\documentclass[12pt]{article}
\usepackage[left=25mm,right=25mm,top=30mm,bottom=30mm]{geometry}
\usepackage{amsmath} % math
\usepackage{amssymb} % math
\usepackage{graphicx} % to use \includegraphics{}
\usepackage{amsthm}
\usepackage{diagbox} % to make tables
\usepackage{kotex} % to use korean(hangul)
\usepackage{caption}
\usepackage{color}
\usepackage{setspace}
\usepackage{tabularx}
\usepackage{xfrac}
\usepackage[firstpage]{draftwatermark}
\usepackage[hidelinks]{hyperref}
\usepackage[group-separator={,}]{siunitx} %si unit
% \usepackage[hangul]{kotex} 이걸로하면 Abstract, Contents 등이 전부 요약, 내용 등으로 바뀜.

\pagenumbering{roman} % \pagenumbering{다른거}나올때까지 roman으로 
%pagenumbering.
\renewcommand\thesection{\Roman{section}} % \section : In roman(I,II,III,...)
\renewcommand\thesubsection{\arabic{subsection}} % \subsection : In 
%arabic(1,2,3,...)

\usepackage{chngcntr}
\renewcommand\thesubsubsection{%
	\thesubsection.\arabic{subsubsection}%
}
\usepackage{titlesec}
\titlelabel{\thetitle.\quad}
%subsubsection에 점이 두 개씩 찍히는 것을 막고,
%점이 문단 번호에만 찍히도록 수정.

\usepackage{tocloft}%
\setlength{\cftsecnumwidth}{3em}% 
\setlength{\cftsubsecnumwidth}{1em}%
\setlength{\cftsubsubsecnumwidth}{2em}%
%toc의 여백 조정으로 VIII와 같은 번호가 제목과 겹치는 것을 방지

\SetWatermarkText{\includegraphics[angle=-45]{gshslatex_v2.jpg}}

\begin{document}
	
	{\bf \Huge 예제로 배우는 {\color{blue}\LaTeX} 문서 작성}
	
	\begin{LARGE}
		입문자를 위한 \LaTeX 문서 작성의 기초
		\newline
		\newline
		\newline
		\newline
		\newline
	\end{LARGE} 

	\begin{flushright}
		\textbf{ \huge {\bf {\color{blue}\LaTeX} by Example}}
		
		\LARGE {\onehalfspacing  AN INTRODUCTION TO }
			
			GENERAL PURPOSE \LaTeX
			
			
		\normalsize	\onehalfspacing ver 0.2.3
	\end{flushright}
	
	\SetWatermarkAngle{0}
	\SetWatermarkText{\includegraphics{gshs-tex-society-icon.jpg}}
	
	\begin{flushright}
		\vfill \LARGE 경기과학고등학교 \LaTeX 사용자 협회 지음
	\end{flushright}
	
	\clearpage
	\tableofcontents
	\clearpage
	\pagenumbering{arabic}
	
	\section{왜 \LaTeX인가? \LaTeX 시작하기}
	\subsection{이번 장의 목표}
	이번 장을 통해 여러분은 다음 사항을 학습할 것이다:
	\begin{itemize}
		\item \LaTeX 설치하기
		\item \LaTeX 란 무엇인가
		\item \LaTeX 를 사용하는 이유
	\end{itemize}
	
	\subsection{\LaTeX 설치하기}
	여러분은 이 책의 1단원 목차 순서를 보고 이상한 기분이 들 것이다. 왜냐하면, 주로 무엇을 소개를 하는 책은 그것이 무엇인지와 그것을 사용하는 이유에 대하여 먼저 설명을 한 다음에 설치를 하는 방법에 대하여 설명하거나 아예 설치를 하는 방법이 아예 나와있지 않는 경우도 많기 때문이다. 하지만 설치방법이 먼저 나온데에는 깊은 뜻이 있다. 검퓨터의 성능에 따라 \LaTeX 를 설치하는데 1시간이 넘게 걸리는 경우도 있기 때문에 설치를 시작하고 \LaTeX 에 대하여 알아보는 것을 권장한다.
	\subsubsection{kotexlive 설치}
	\TeX Live는 \TeX 의 문서작성 시스템 외에 여러가지 \TeX 관련 프로그램, 패키지, 매크로, 등을 포함하고 있으며, \TeX 사용 환경을 조성해주는데에 사용이 된다. 한글을 입력하기 위해서는 kotexlive가 필요하며 kotexlive는 아래 주소를 통해 다운 받을 수 있다.
	\newline{\bf \url{www.ktug.org/xe/install}}
	\subsubsection{Editor 설치}
	\TeX 를 사용하기 위해서는 Editor가 필요하다. Microsoft Windows와 같은 경우에는 TeXworks, TeXstudio, TeXnicCenter와 같은 여러 무료 프로그램이 존재한다.\TeX 를 사용했던 경기과학고등학교 선배들은 TeXstudio 사용을 권장했다. TeXstudio는 아래 주소를 통해 다운 받을 수 있다.
	\newline{\bf \url{www.texstudio.org/}}
	\subsection{\LaTeX 란 무엇인가}
	자, 이제 드디어 여러분이 궁금한 \LaTeX 가 무엇인지 알아보는 시간이다. \LaTeX 는 텍스트 및 문서의 구조와 의미를 나타내는 {\bf 명령어}를 작성하고 \LaTeX 프로그렘을 통하여 깔끔한 형태의 문서를 만드는 것이다. 즉, \LaTeX 는 마이크로소프트 워드와 같이 {\bf 어떻게 보이는지} WYSIWYG(what you see is what you get)에 중점을 두지 않고 HTML와 같이 {\bf 어떤 것인지} WYSIWYG(what you see is what you mean)에 중점을 둔다.
	\subsection{\LaTeX 를 사용하는 이유}
	\LaTeX 를 사용하는 데에는 여러가지 장점이 있다
	\begin{itemize}
		\item 수식편집기로써는 수학에서 표준으로 쓰인다.
		\item 초기 설정을 잘 해두면 작업량이 줄어들게 된다.
		\begin{itemize}
			\item 차례, 그림/표 목차를 명령어 하나로 추가 할 수 있다.
			\item 참고문헌을 자동으로 인용순으로 정열을 할 수 있다.
		\end{itemize}
		\item Cross-referencing 을 쉽게 사용을 할 수가 있다.
		\item 벡터 이미지(svg, eps, pdf)를 손쉽게 첨부를 할 수가 있다.
	\end{itemize}
	하지만 어떠한 것도 장저만 있을 수는 없다. \LaTeX 의 단점으로는
	\begin{itemize}
		\item 처음 사용하는 사람이 읽거나 편집하기가 어렵다
		\item 양식을 초기 설정을 하기가 어렵다
		\item 실시간으로 편집을 하고 있는게 보이지 않는다
	\end{itemize}
	위의 단점에서 알 수 있듯이 \LaTeX 를 처음 사용하는 여러분은 초기설정이나 양식을 맞추는데에 어려움이 있을 수 있다. 이 때문에 {\bf 경기과학고등학교 \LaTeX 사용자 협회}가 설립 되었으며  여러분을 위해 경기과학고등학교에서 생활 하면서 사용하게 될 \LaTeX 양식과 학습자료, 그리고 예시로 작성이 된 여러가지 문서들을 겅기과학고등학교 \LaTeX 사용자 협회 홈페이지({\bf latex.gs.hs.kr})예서 제공한다(참고로 이 책도 \LaTeX 로 작성이 되었으며 모든 학습자료 역시 \LaTeX 로 작성이 되어 있다).
	\clearpage
	
	\section{\LaTeX 와 친해지기}
	\subsection{이번장의 목표}
	이번 장에서 여러분은 다음 사항을 학습 할 것이다.
	\begin{itemize}
		\item \LaTeX 의 문서 구조
		\item \LaTeX 글꼴 설정
		\item \LaTeX 단락 설정
	\end{itemize}
	\subsection{\LaTeX 문서구조}
	\LaTeX 문서는 크게 두 부분으로 나누어 지게 된다
	\subsubsection{Preamble(프림블)}
	Preamble은 문서의 유형(documentclass) 들을 선언하며, 필요한 환경설정 및 명령어 역시 여기서 선언을 한다. 프로그레밍의 Header와 같은 역활을 하게 된다.
	\subsubsection{Body(본문)}
	본문에서는 예상을 할 수가 있듯이 문서의 내용, 문서의 글꼴과 같은 완성이 된 문서의 내용 및 모양을 작성하는 부분이다. 경기과학고등학교  \LaTeX 사용자 협회 일반 부원이라면 주어진 양식으로 주로 작업을 하는 부분이 본문일 것이다.
	\newline
	\newline
	{\bf 이 책의 예시에서는 Preamble와 Body를 //... 로 구분을 해 놓았다}
	\clearpage
	\subsection{Hello World!!}
	자 이젠 여러분은 \LaTeX 를 사용하는 이유, 장점 문서구조에 대하여 배우고 설치 까지 완료를 했으면 이젠 실제로 문서를 작성할 시간이 되었다. \TeX Editor을 키고 새 문서를 만들어 보자. 우리는 이젠 코딩의 관례에 따라 Hello World!!가 쓰인 문서를 작성해 볼 것이다.\newline
	
		\begin{tabularx}{\textwidth \onehalfspacing}{ |X|X| }		
			\hline
			\textbackslash documentclass\{article\}&Hello World!!\\
			\ \ \ \textbackslash begin\{document\}		& \\
			\ \ \ \ \ \ Hello World!!			& \\
			\ \ \ \textbackslash end\{document\}			& \\
			\hline
		\end{tabularx}
		\newline
		\newline
	{\bf 이 책에서의 예시는 따로 언급이 없다면 왼쪽에는 \LaTeX 코드를 오른쪽에는 컴파일 결과를 표시 하겠다.}\newline
	\underline{만약 {\bf 한국어}를 입력하고 싶다면 프림블에 `\textbackslash usepackage\{kotex\}'를 입력하고 사용하면 된다.}\newline\newline
	여러분은 이제 Hello World!!가 적힌 문서를 \LaTeX 를 통하여 만들 수 있게 되었다. 하지만 여러분은 이 Hello World!! 가 너무 평범해 보인다는 것에 동의를 할 것이다. 우리는 여러분의 기대를 저버리지 않기 위해 이제 Hello World!!가 다양한 글꼴을 가지게 할 것이다.
	\newline\newline
	아래는 글자를 {\it italic}, {\bf bold}로 작성을 하거나 \underline{밑줄}치는 예시이다.
	\newline
	{\bf 이 예제부터는 따로 페키지를 사용하지 않는 경우 본문 안의 코드만 표기 할 것이다}\newline
	
	\begin{tabularx}{\textwidth \onehalfspacing}{ |X|X| }
		\hline
		\textbackslash textit\{Hello World!!\} & \textit{Hello World!!}\\
		\{\textbackslash it\ Hello World!! \} & {\it Hello World!!}\\
		\hline
		\textbackslash textbf\{Hello World!!\} & \textbf{Hello World!!}\\
		\{\textbackslash bf\ Hello World!! \} & {\bf Hello World!!}\\
		\hline
		\textbackslash underline\{Hello World!!\} & \underline{Hello World!!}\\
		\hline
	\end{tabularx}
	\newline
	\newline
	눈치 빠른 여러분들은 알아 차렸듯이 italic이나 bold 체는 사용을 할 수 있는 방법이 두 가지가 있다.여러분이 사용하기 편한데로 사용하면 된다.
	\clearpage
	글자를 italic, bold 그리고 밑줄까지 쳤으면 이젠 문자의 크기를 조절 할 차례이다.\newline
	
	\begin{tabularx}{\textwidth \onehalfspacing}{|X|X|}
		\hline
		\textbackslash tiny Hello World!! & \tiny Hello World!!\\
		\textbackslash scriptsize Hello World!! & \scriptsize Hello World!!\\
		\textbackslash footnotesize Hello World!! & \footnotesize Hello World!!\\
		\textbackslash small Hello World!! & \small Hello World!!\\
		\textbackslash normalsize Hello World!! & \normalsize Hello World!!\\
		\textbackslash large Hello World!! & \large Hello World!!\\
		\textbackslash Large Hello World!! & \Large Hello World!!\\
		\textbackslash LARGE Hello World!! & \LARGE Hello World!!\\
		\textbackslash huge Hello World!! & \huge Hello World!!\\
		\textbackslash Huge Hello World!! & \Huge Hello World!!\\
		\hline
	\end{tabularx}
	\newline
	\newline
	위와 같은 방법을 사용하면 \textbackslash size 뒤에 있는 {\bf 모든 턱스트 크기가} 변하게 된다. 만약 {\large 이것처럼} 특정 부분에만 글자 크기를 적용 하고 싶다면 \newline
	
	\begin{tabularx}{\textwidth \doublespacing}{ |X|X| }
		\hline
		\{\textbackslash LARGE LARGE\} normal & {\LARGE LARGE} normal\\
		\hline
	\end{tabularx}
	\newline
	\newline
	와 같이 사용하면 된다.
	\newline
	
	이젠 특수부호를 입력하는 방법을 배울 시간이다. WYSIWYG인 워드 프로세서와 달리 \LaTeX 와 같은 WYSIWYM인 문서를 작성을 할때에는 많은 특수기호들이 명령어와 겹치기 때문에 일반적인 방법으로 특수 문자들을 입력을 해줄 수 없다.몇가지 특수기호들의 예시를 들어 보겠다.\newline
	
	\begin{tabularx}{\textwidth \onehalfspacing}{|X|X|}
		\hline
		\textbackslash \%, \textbackslash \&, \textbackslash \_, \textbackslash \$, \textbackslash \#, \textbackslash \{, \textbackslash\} & \%, \&, \_, \$, \#,\{, \}\\
		\hline
	\end{tabularx}
	\newline
	\newline
	축하한다! 여러분은 이젠 기본적인 문장을 \LaTeX 로 작성 할 수 있게 되었다.
	\clearpage
	
	\subsection{\LaTeX 단락 설정하기}
	\subsubsection{이 단원의 목표}
	이번 장에서 여러분은 다음 사항을 학습 할 것이다.
	\begin{itemize}
		\item 개행하기
		\item 좌측, 우측, 중간 정렬
		\item 아래 정렬
		\item 줄간격 조절
		\item 주석, 메모, 코멘트
		\item Section 나누기
	\end{itemize} 
	\subsubsection{개행하기}
	개행하는 방법에는 총 3가지가 있으며 2가지 결과가 나오게 된다.\newline
	
	\begin{tabularx}{\textwidth \onehalfspacing}{|X|X|}
		\hline
		Hello World!! & Hello World!!\\
		\ & \ \ \ \ \ \ Hello World!!\\
		Hello World!! & \\
		\hline
		Hello World!!\textbackslash newline Hello World!! & Hello World!!\\
		\ &Hello World!!\\
		\hline
		Hello World!! \textbackslash \textbackslash Hello World!! & Hello World!!\\
		\ & Hello World!!\\
		\hline
	\end{tabularx}
	\newline
	\newline
	위에서 볼 수 있듯이 두번 개행할 경우 새로운 문단으로 시작하게 되며 다른 두 방법은 줄만 바뀌게 된다.
	\subsubsection{좌측, 우측, 중간 정렬}
	좌측, 우측, 중간 정렬은 적용할 글자가 적을 경우에는 다음과 같은 방법을 사용하면 된다.\newline
	
	\begin{tabularx}{\textwidth \onehalfspacing}{|X|X|}
		\hline
		\{\textbackslash flushleft Hello World!! \} & Hello World!!\\
		\hline
		\{\textbackslash flushright Hello World!! \} & \multicolumn{1}{|r|}{Hello World!!}\\
		\hline
		\{\textbackslash centering Hello World!! \} & \multicolumn{1}{|c|}{Hello World!!}\\
		\hline
	\end{tabularx}
	\clearpage
	적용할 글자가 한 문단이거나 많을 경우에는 다음과 같은 방법을 사용하는 것을 추천한다. \newline
	
	\begin{tabularx}{\textwidth \onehalfspacing}{|X|X|}
		\hline
		\textbackslash begin\{flushleft\} & Hello World!!\\
		\ \ \ \ \ \ Hello World!! & \\
		\textbackslash end\{flushleft\} & \\
		\hline
		\textbackslash begin\{flushright\} & \multicolumn{1}{|r|}{Hello World!!}\\
		\ \ \ \ \ \ Hello World!! & \\
		\textbackslash end\{flushright\} & \\
		\hline
		\textbackslash begin\{center\} & \multicolumn{1}{|c|}{Hello World!!}\\
		\ \ \ \ \ \ Hello World!! & \\
		\textbackslash end\{center\} & \\
		\hline
	\end{tabularx}
	\subsubsection{아래정렬}
	이 책의 표지와 같이 글을 페이지 아래에 적어야 할때도 있다. 그럴 경우 아래와 같은 방법을 사용하면 된다.\newline
	
	\begin{tabularx}{\textwidth \onehalfspacing}{|X|X|}
		\hline
		\textbackslash vfill Hello World!! & \\
		 & \\
		 & Hello World!!\\
		\hline
	\end{tabularx}
	\subsubsection{줄간격 조절하기}
	우리는 가독성을 높이기 위해서 줄간격을 조절 할 때가 있다. 여러분은 크게 2가지 방식으로 줄간격을 조절 할 수가 있다.\newline
	
	\begin{tabularx}{\textwidth}{|X|X|}
		\hline
		Contents... &The quick brown fox jumps over the lazy dog\\
		\hline
		\textbackslash usepackage\{setspace\} 
		
		//...
		
		\textbackslash onehalfspacing Contents... & \onehalfspacing The quick brown fox jumps over the lazy dog\\
		\hline
		\textbackslash usepackage\{setspace\} 
		
		//...
		
		\textbackslash  doublespacing Contents... & \doublespacing The quick brown fox jumps over the lazy dog\\
		\hline
	\end{tabularx}
	\newline
	\newline
	위와 같이 사용 할 경우 \textbackslash \_\_\_\_spacing 뒤의 모든 텍스트 들에 적용이 됨으로 예를 들어 한 문단에만 적용을 하고 싶으면 다음과 같이 사용하면 된다.\newline
	
	\begin{tabularx}{\textwidth}{|X|X|}
		\hline
		Contents... & The quick brown fox jumps over the lazy dog\\
		\hline
		\textbackslash usepackage\{setspace\} 
		
		//...
		
		\textbackslash \{spacing\}\{1.5\}
		
		Contents...
		
		\textbackslash \{spacing\} & \begin{spacing}{1.5} 
			The quick brown fox jumps over the lazy dog
		\end{spacing}\\
		\hline
	\end{tabularx}
	\subsubsection{주석, 메모, 코멘트}
	코드를 작성할 때 여러분은 주석 처리 하거나, 또는 문서 작성을 할 때 메모나 코멘트를 해야 하는 경우가 있었을 것이다.\newline
	주석 처리를 하기 위해서는 다음과 같은 방법을 사용한다.\newline
	
	\begin{tabularx}{\textwidth \onehalfspacing}{|X|X|}
		\hline
		visible dog & visible dog\\
		\% invisible cat & \\
		\hline
	\end{tabularx}
	\newline
	\newline
	많은 글을 주석 처리를 해야 할 경우에는 다음과 같은 방법을 사용 하기를 권장한다.\newline
	
	\begin{tabularx}{\textwidth \onehalfspacing}{|X|X|}
		\hline
		\textbackslash usepackage\{verbatim\} & visible dog\\
		//...& \\
		visible dog & \\
		\textbackslash begin\{comment\} & \\
		\ \ \ \ \ \ invisible cat & \\
		\textbackslash end\{comment\} & \\
		\hline
	\end{tabularx}
	\newline
	\newline
	그 외에도 여러가지 주석처리나 메모기능이 있으며 만약 다른 방법을 사용하고 싶다면 인터넷에서 찾아보기 바란다.
	\subsubsection{Section 나누기}
	논문이나 보고서를 작성을 할 때 이 책 처럼 문서를 장, 단원, 등으로 나누어야 할 때가 있다. \LaTeX 를 사용하면 손 쉽게 Section 나누기를 할 수가 있다.\LaTeX 특성상 표 안에 예시 결과를 넣을 수 없기 떄문에 2장의 Section을 예시로 하겠다.\newline
	
	\begin{tabularx}{\textwidth \onehalfspacing}{|X|}
		\hline
		\textbackslash section\{\LaTeX 와 친해지기 \}\\
		\textbackslash subsection\{이번장의 목표\}\\
		\textbackslash subsection\{\LaTeX 의 문서구조 \}\\
		\textbackslash subsubsection\{Preamble\}\\
		\textbackslash subsubsection\{Body\}\\
		\hline
	\end{tabularx}
	\newline
	\newline
	와 같이 사용하면 2장의 2.2까지 Section으로 나눌 수가 있다. 내용은 Section 사이사이에 들어간다.
	\clearpage
	
	\section{수식에 특화된 \LaTeX}
	\subsection{이번 장의 목표}
	\begin{itemize}
		\item 수식 입력하기
		\item 기호 입력하기
		\item 연산자 입력하기
		\item 액센트 입력
	\end{itemize}
	\subsection{수식 입력하기}
	\LaTeX 로 논문이나 보고서를 작성 할 때 수식을 넣어야 하는 경우가 많을 것이다. 수식을 입력 할때 다음과 같은 환경을 만들어 주어야 한다.\newline
	
	\begin{tabularx}{\textwidth \onehalfspacing}{|X|X|}
		\hline
		Equation: \textbackslash (1+b=3\textbackslash ) & Equation: \(1+b=3\)\\
		\hline
		Equation: \textbackslash[1+b=3\textbackslash] & Equation: \[1+b=3\]\\
		\hline
		Equation: \$1+b=3\$ & Equation: $1+b=3$\\
		\hline
		Equation: \$\$1+b=3\$\$ & Equation: $$1+b=3$$\\
		\hline
		Equation:
		\textbackslash begin\{math\}
		
		1+b=3
		
		\textbackslash end\{math\} & 
		Equation:
		\begin{math}
		1+b=3
		\end{math}\\
		\hline
		Equation:
		\textbackslash begin\{displaymath\}
		
		1+b=3
		
		\textbackslash end\{displaymath\} & 
		Equation:
		\begin{displaymath}
		1+b=3
		\end{displaymath}\\
		\hline
	\end{tabularx}
	\newline
	\newline
	홀수번째와 짝수번째의 차이는 수식을 텍스트와 같이 있느냐 아니면 따로 디스플래이 하느냐에 있다. \newline
	{\bf 주의: \$\$...\$\$는 AMS-\LaTeX 같은 매크로와 충돌이 생겨 문제가 생길 수 있으므로 사용을 지양해야 한다.}\clearpage
	\subsection{기호 입력하기}
	수학에서는 여러가지 기호가 사용이 된다. 여러분은 그 여러가지 기호를 수식에도 입력을 해야 하기 때문에 이 단원에는 여러가지 기호의 예시를 보여 주도록 하겠다.\newline\newline
	처음으로 바로 사용할 수 있는 기호목록이다.\newline
	{\bf 이 장의 예제는 이제부터 예시는 수식 환경 안에 있는 내용만 표기 하겠다.}\newline
	
	\begin{tabularx}{\textwidth \onehalfspacing}{|X|}
		\hline
		\(+ - / = ! ( ) [ ] < > | ' :\)\\
		\hline
	\end{tabularx}
	\newline
	\newline
	그리스 문자들은 다음과 같이 사용하면 된다.\newline
	
	\begin{tabularx}{\textwidth \onehalfspacing}{|X|X|}
		\hline
		\textbackslash alpha, \textbackslash beta, \textbackslash gamma, \textbackslash pi, \textbackslash phi, \textbackslash varphi & \(\alpha \beta \gamma \pi \phi \varphi\)\\
		\hline
	\end{tabularx}
	\newline\newline
	다음은 삼각함수 사용의 예시이다.\newline
	
	\begin{tabularx}{\textwidth \onehalfspacing}{|X|X||X|X||X|X||X|X|}
		\hline
		수식&코드&수식&코드&수식&코드&수식&코드\\
		\hline
		\hline
		\(\sin\)&\textbackslash sin &
		\(\cos\)&\textbackslash cos &
		\(\tan\)&\textbackslash tan &
		\(\cot\)&\textbackslash cot \\
		\hline
		\(\arcsin\)&\textbackslash arcsin &
		\(\arccos\)&\textbackslash arccos &
		\(\arctan\)&\textbackslash arctan &
		&\\
		\hline
		\(\sinh\)&\textbackslash sinh &
		\(\cosh\)&\textbackslash cosh &
		\(\tanh\)&\textbackslash tanh &
		\(\coth\)&\textbackslash coth \\
		\hline
		\(\sec\)&\textbackslash sec &
		\(\csc\)&\textbackslash csc &
		 & & & \\
		 \hline
	\end{tabularx}
	\newline
	\newline
	다음은 논리 기호 입력 방법이다.\newline
	
	\begin{tabularx}{\textwidth \onehalfspacing}{|l|X||l|X||l|X||l|X|}
		\hline
		기호&코드&기호&코드&기호&코드&기호&코드\\
		\hline
		\hline
		\(\exists\)&\textbackslash exists&
		\(\nexists\)&\textbackslash nexists&
		\(\forall\)&\textbackslash forall&
		&\\
		\hline
		\(\subset\)&\textbackslash subset&
		\(\supset\)&\textbackslash supset&
		\(\in\)&\textbackslash in&
		\(\ni\)&\textbackslash ni\\
		\hline
		\(\notin\)&\textbackslash notin &
		\(\emptyset \)&\textbackslash emptyset &
		\(\top\)&\textbackslash top &
		\(\bot\)&\textbackslash bot \\
		\hline
		\(\rightarrow\)&\textbackslash rightarrow &
		\(\leftarrow\)&\textbackslash leftarrow &
		\(\Rightarrow\)&\textbackslash Rightarrow &
		\(\Leftarrow\)&\textbackslash Leftarrow \\
		\hline
		\(\leftrightarrow\)&\small\textbackslash leftrightarrow &
		\(\Leftrightarrow \)&\small\textbackslash Leftrightarrow &
		\(\land\)&\textbackslash land &
		\(\lor\)&\textbackslash lor \\
		\hline
	\end{tabularx}
	\clearpage
	다음은 관계기호 입력 방법이다.\newline
	
	\begin{tabularx}{\textwidth \onehalfspacing}{|l|X||l|X||l|X||l|X|}
		\hline
		기호&코드&기호&코드&기호&코드&기호&코드\\
		\hline
		\hline
		\(\leq\)&\textbackslash leq &
		\(\geq\)&\textbackslash geq &
		\(\ll\)&\textbackslash ll &
		\(\gg\)&\textbackslash gg \\
		\hline
		\(\subset\)&\textbackslash subset &
		\(\supset\)&\textbackslash supset &
		\(\subseteq\)&\textbackslash subseteq &
		\(\supseteq\)&\textbackslash supseteq \\
		\hline
		\(\nsubseteq\)&\textbackslash nsubseteq &
		\(\nsupseteq\)&\textbackslash nsupseteq &
		\(\parallel\)&\textbackslash parallel &
		\(\nparallel\)&\textbackslash nparallel \\
		\hline
		\(\prec\)&\textbackslash prec &
		\(\succ\)&\textbackslash succ &
		\(\preceq\)&\textbackslash preceq &
		\(\succeq\)&\textbackslash succeq \\
		\hline
		\(\doteq\)&\textbackslash doteq &
		\(\equiv\)&\textbackslash equiv &
		\(\approx\)&\textbackslash approx &
		\(\cong\)&\textbackslash cong \\
		\hline
		\(\sim\)&\textbackslash sim &
		\(\simeq\)&\textbackslash simeq &
		\(\neq\)&\textbackslash neq &
		\(\propto\)&\textbackslash propto \\
		\hline
	\end{tabularx}
	\newline\newline
	여기에 없는 다른 기호들은 인터넷에서 찾아보거나 \TeX studio를 사용할 경우 원쪽 바에 수식 아카이브가 있으므로 여러분은 유용하게 사용하기만 하면 된다.
	\subsection{연산자 입력하기}
	여러분은 \LaTeX 에서 여러가지 수학 기호를 배웠다. 이젠 배운 수학기호들은 사용 할 때가 왔다. 다음은 연산자를 이용한 수식입력의 예시이다.\newline
	
	\begin{tabularx}{\textwidth \large \onehalfspacing}{|X|X|}
		\hline
		k\_1+k\textasciicircum2=3 & \(k_1+k^2=3\)\\
		\hline
		\textbackslash lim\_\{x \textbackslash to\ \textbackslash infty\}x=\textbackslash infty&\(\lim_{x \to \infty}x=\infty \)\\
		\hline
		\textbackslash exp(a)=1&\(\exp(a)=1\)\\
		\hline
		\textbackslash cos(3\textbackslash theta)=\textbackslash cos\textasciicircum3 \textbackslash theta & \(\cos(3\theta)=\cos^3\theta\)\\
		\hline
		x \textbackslash equiv 3 \textbackslash pmod\{4\}&\(x \equiv 3 \pmod{4}\)\\
		\hline
		\textbackslash frac\{a\}\{4\} & \(\frac{a}{4} \)\\
		\hline
		\textbackslash displaystyle \textbackslash sum\_\{k=0\}\textasciicircum\{10\}x&\(\displaystyle \sum_{k=0}^{10}x \)\\
		\hline
		\textbackslash int\_0\textasciicircum \textbackslash infty x\textbackslash ,\textbackslash mathrm\{d\}x& \(\int_{0}^{\infty}x\,\mathrm{d}x \)\\
		\hline
		\textbackslash frac\{\textbackslash mathrm d\}\{\textbackslash mathrm d x\} \textbackslash big( k f(x) \textbackslash big)&\(\frac{\mathrm{d}}{\mathrm d x}\big( k g(x) \big) \)\\
		\hline
		x \textbackslash in [-2,1]&\(x\in [-2,1] \)\\
		\hline
		\small\textbackslash begin \{bmatrix\}\newline
		1\&0\&1 \textbackslash\textbackslash
		0\&1\&0 \newline
		\textbackslash end\{bmatrix\}&\ \small\newline
		\(\begin{bmatrix}
		1&0&1\\0&1&0
		\end{bmatrix} \)\\
		\hline
	\end{tabularx}
	\newline\newline
	여기에 없는 다른 수식/연산자 들은 인터넷에서 찾아보자.
	\subsection{엑센트 입력하기}
	수식을 입력할 때에 벡터와 같이 엑센트를 사용 하는 경우가 적지 않게 있다. 엑센트를 사용하는 예시는 다음과 같다.\newline
	
	\begin{tabularx}{\textwidth \onehalfspacing}{|l|X||l|X|}
		\hline
		\(a'\) & a' & \(a''\) & a''\\
		\hline
		\(\hat{a} \) & \textbackslash hat\{a\} & \(\bar{a} \) & \textbackslash bar \{a\}\\
		\hline
		\(\dot{a}\) & \textbackslash dot\{a\} & \(\ddot{a}\) &\textbackslash ddot\{a\}\\
		\hline
		\(\overrightarrow{AB}\) & \textbackslash overrightarrow\{AB\} & \(\overleftarrow{AB}\) & \textbackslash overleftarrow\{AB\}\\
		\hline
		\(\overline{abc}\) & \textbackslash overline\{abc\}& \(\vec{v}\) & \textbackslash vec\{v\} \\
		\hline
	\end{tabularx} 
	\section{여러가지 요소삽입}
	\subsection{이번장의 목표}
	\begin{itemize}
		\item 목차 삽입
		\item 아이템 열거 삽입
		\item 이미지 삽입하기
		\item 표 삽입하기
		\item 머리말/꼬리말 삽입
	\end{itemize}
	\subsection{목차 삽입하기}
	책이나 논문은 많은 section으로 나위어 지는 경우가 많으며 이 section의 페이지나 대략적인 내용을 알려주기 위하여 목차를 넣는 경우가 많다. \LaTeX 에서는 간단하게 한 스크립트만 입력을 해주면 목차를 작성해 주게 된다.\newline
	
	\begin{tabularx}{\textwidth \onehalfspacing}{|X|}
		\hline
		\textbackslash tableofcontents\\
		\hline
	\end{tabularx}
	\newline\newline
	\textbf{주의: }목차는 두번 컴파일 해 주어야 업데이트가 된다.\\
	다음은 그림, 표 목록과 켭션을 출력하기 위해서는 다음 스크립트를 입려해 주면 된다.\newline
	
	\begin{tabularx}{\textwidth \onehalfspacing}{|X|}
		\hline
		\textbackslash listoffigures\\
		\textbackslash listoftables\\
		\hline
	\end{tabularx}
	\clearpage
	\subsection{각주 삽입하기}
	각주 입력하기는 매우 쉽다.\\
	\begin{tabularx}{\textwidth \onehalfspacing}{|X|X|}
		\hline
		Where is footnote?\textbackslash footnote\{Here it is!\}&Where is footnote? \footnote{Here it is!}\\
		\hline
	\end{tabularx}
	\newline\newline
	
	\subsection{아이템 열거 삽입하기}
	\subsection{이미지 삽입하기}
	\subsection{표 삽입하기}
	\subsection{머리말/꼬리말 삽입하기}
	\section{Advanced Mathematics}
	\subsection{Equation numbering}
	\subsubsection{Equation numbering}
	\subsubsection{Equation Referencing}
	\subsection{Align Equation}
	\subsubsection{Align}
	\subsubsection{Aligned braces}
	\subsection{Formatting Equations}
	\subsubsection{color}
	\subsubsection{Fonts}
	\section{Advanced use of \LaTeX}
	\subsection{customize marking}
	\subsubsection{footnote}
	\subsubsection{section}
	\subsection{referencing}
	\subsection{algorithms}
	\subsection{source code}
\end{document}
