\documentclass[12pt]{article}
\usepackage[left=25mm,right=25mm,top=30mm,bottom=30mm]{geometry}
\usepackage{amsmath} % math
\usepackage{amssymb} % math
\usepackage{graphicx} % to use \includegraphics{}
\usepackage{amsthm}
\usepackage{diagbox} % to make tables
\usepackage{kotex} % to use korean(hangul)
\usepackage{caption}
\usepackage{color}
\usepackage{setspace}
\usepackage{tabularx}
\usepackage{xfrac}
\usepackage{tikz}
\usepackage{amsfonts}
\usepackage{listings}
\usepackage{mathrsfs}
\usepackage[firstpage]{draftwatermark}
\usepackage[hidelinks]{hyperref}
\usepackage[]{algorithm2e}
\usepackage[group-separator={,}]{siunitx} %si unit

% \usepackage[hangul]{kotex} 이걸로하면 Abstract, Contents 등이 전부 요약, 내용 등으로 바뀜.

\pagenumbering{roman} % \pagenumbering{다른거}나올때까지 roman으로 
%pagenumbering.
\renewcommand\thesection{\Roman{section}} % \section : In roman(I,II,III,...)
\renewcommand\thesubsection{\arabic{subsection}} % \subsection : In 
%arabic(1,2,3,...)

\usepackage{chngcntr}
\renewcommand\thesubsubsection{%
	\thesubsection.\arabic{subsubsection}%
}
\usepackage{titlesec}
\titlelabel{\thetitle.\quad}
%subsubsection에 점이 두 개씩 찍히는 것을 막고,
%점이 문단 번호에만 찍히도록 수정.

\usepackage{tocloft}%
\setlength{\cftsecnumwidth}{3em}% 
\setlength{\cftsubsecnumwidth}{1em}%
\setlength{\cftsubsubsecnumwidth}{2em}%
%toc의 여백 조정으로 VIII와 같은 번호가 제목과 겹치는 것을 방지

\SetWatermarkText{\includegraphics[angle=-45]{gshslatex_v2.jpg}}

\begin{document}
	
	{\bf \Huge 예제로 배우는 {\color{blue}\LaTeX} 문서 작성}
	
	\begin{LARGE}
		입문자를 위한 \LaTeX 문서 작성의 기초
		\newline
		\newline
		\newline
		\newline
		\newline
	\end{LARGE} 

	\begin{flushright}
		\textbf{ \huge {\bf {\color{blue}\LaTeX} by Example}}
		
		\LARGE {\onehalfspacing  AN INTRODUCTION TO }
			
			GENERAL PURPOSE \LaTeX
			
			
		\normalsize	\onehalfspacing ver 1.0.0
	\end{flushright}
	
	\SetWatermarkAngle{0}
	\SetWatermarkText{\includegraphics{gshs-tex-society-icon.jpg}}
	
	\begin{flushright}
		\vfill \LARGE 경기과학고등학교 \LaTeX 사용자 협회 지음
	\end{flushright}
	
	\clearpage
	\tableofcontents
	\clearpage
	\pagenumbering{arabic}
	
	\section{왜 \LaTeX인가? \LaTeX 시작하기}
	\subsection{이번 장의 목표}
	이번 장을 통해 여러분은 다음 사항을 학습할 것이다:
	\begin{itemize}
		\item \LaTeX 설치하기
		\item \LaTeX 란 무엇인가
		\item \LaTeX 를 사용하는 이유
	\end{itemize}
	
	\subsection{\LaTeX 설치하기}
	여러분은 이 책의 1단원 목차 순서를 보고 이상한 기분이 들 것이다. 왜냐하면, 주로 무엇에 대해 소개를 하는 책은 그것이 무엇인지와 그것을 사용하는 이유에 대하여 먼저 설명을 한 다음에 설치를 하는 방법에 대하여 설명하거나 아예 설치를 하는 방법이 아예 나와있지 않는 경우도 많기 때문이다. 하지만 설치방법이 먼저 나온데에는 깊은 뜻이 있다. 검퓨터의 성능에 따라 \LaTeX 를 설치하는데 1시간이 넘게 걸리는 경우도 있기 때문에 설치를 시작하고 \LaTeX 에 대하여 알아보는 것을 권장한다.
	\subsubsection{kotexlive 설치}
	\TeX Live는 \TeX 의 문서작성 시스템 외에 여러가지 \TeX 관련 프로그램, 패키지, 매크로, 등을 포함하고 있으며, \TeX\ 사용 환경을 조성해주는데에 사용이 된다. 한글을 입력하기 위해서는 kotexlive가 필요하며 kotexlive는 아래 주소에서 다운 받을 수 있다.
	\newline{\bf \url{www.ktug.org/xe/install}}
	\subsubsection{Editor 설치}
	\TeX 를 사용하기 위해서는 Editor가 필요하다. Microsoft Windows와 같은 경우에는 TeXworks, TeXstudio, TeXnicCenter와 같은 여러 무료 프로그램이 존재한다.\TeX 를 사용했던 경기과학고등학교 선배들은 TeXstudio 사용을 권장했다. TeXstudio는 아래 주소를 통해 다운 받을 수 있다.
	\newline{\bf \url{www.texstudio.org/}}
	\subsection{\LaTeX 란 무엇인가}
	자, 이제 드디어 여러분이 궁금한 \LaTeX 가 무엇인지 알아보는 시간이다. \LaTeX 는 텍스트 및 문서의 구조와 의미를 나타내는 {\bf 명령어}를 작성하고 \LaTeX 프로그렘을 통하여 깔끔한 형태의 문서를 만드는 것이다. 즉, \LaTeX 는 마이크로소프트 워드와 같이 {\bf 어떻게 보이는지} WYSIWYG(what you see is what you get)에 중점을 두지 않고 HTML와 같이 {\bf 어떤 것인지} WYSIWYM(what you see is what you mean)에 중점을 둔다.
	\subsection{\LaTeX 를 사용하는 이유}
	\LaTeX 를 사용하는 데에는 여러가지 장점이 있다
	\begin{itemize}
		\item 수식편집기로써는 수학에서 표준으로 쓰인다.
		\item 초기 설정을 잘 해두면 작업량이 줄어들게 된다.
		\begin{itemize}
			\item 차례, 그림/표 목차를 명령어 하나로 추가할 수 있다.
			\item 참고문헌을 자동으로 인용순으로 정렬 할 수 있다.
		\end{itemize}
		\item Cross-referencing 을 쉽게 할 수가 있다.
		\item 벡터 이미지(svg, eps, pdf)를 손쉽게 첨부를 할 수가 있다.
	\end{itemize}
	하지만 어떠한 것도 장점만 있을 수는 없다. \LaTeX 의 단점으로는
	\begin{itemize}
		\item 처음 사용하는 사람이 읽거나 편집하기가 어렵다
		\item 양식을 초기 설정을 하기가 어렵다
		\item 실시간으로 편집을 하고 있는게 보이지 않는다
	\end{itemize}
	위의 단점에서 알 수 있듯이 \LaTeX 를 처음 사용하는 여러분은 초기설정이나 양식을 맞추는데에 어려움이 있을 수 있다. 이 때문에 {\bf 경기과학고등학교 \LaTeX 사용자 협회}가 설립 되었으며  여러분을 위해 경기과학고등학교에서 생활 하면서 사용하게 될 \LaTeX 양식과 학습자료, 그리고 예시로 작성이 된 여러가지 문서들을 겅기과학고등학교 \LaTeX 사용자 협회 홈페이지({\bf latex.gs.hs.kr})예서 제공한다(참고로 이 책도 \LaTeX 로 작성이 되었으며 모든 학습자료 역시 \LaTeX 로 작성이 되어 있다).
	\clearpage
	
	\section{\LaTeX 와 친해지기}
	\subsection{이번장의 목표}
	이번 장에서 여러분은 다음 사항을 학습 할 것이다.
	\begin{itemize}
		\item \LaTeX 의 문서 구조
		\item \LaTeX 글꼴 설정
		\item \LaTeX 단락 설정
	\end{itemize}
	\subsection{\LaTeX 문서구조}
	\LaTeX 문서는 크게 두 부분으로 나누어 지게 된다
	\subsubsection{Preamble(프림블)}
	Preamble은 문서의 유형(documentclass) 들을 선언하며, 필요한 환경설정 및 명령어 역시 여기서 선언을 한다. 프로그래밍의 Header와 같은 역활을 하게 된다.
	\subsubsection{Body(본문)}
	본문에서는 예상을 할 수가 있듯이 문서의 내용, 문서의 글꼴과 같은 완성이 된 문서의 내용 및 모양을 작성하는 부분이다. 경기과학고등학교  \LaTeX 사용자 협회 일반 부원이라면 주어진 양식으로 주로 작업을 하는 부분이 본문일 것이다.
	\newline
	\newline
	{\bf 이 책의 예시에서는 Preamble와 Body를 //... 로 구분을 해 놓았다}
	\subsection{용어 설명}
	이 책을 읽으면서 보게될 용어를 설명 해 주려 한다.
	
	처음으로 \textbf{환경}이다. 환경은 환경안의 텍스트가 어떠한 의미를 가질지 정의하는데 사용이 된다.예를 들어 \textbf{\textbackslash begin\{asdf\}}와 \textbf{\textbackslash end\{asdf\}}안에 있는 텍스트는 \textbf{``asdf 환경 안에 있다''} 라고 표현한다. 환경에는 글꼴을 나타내는 환경, 표를 나타내는 환경, 단락을 나타내는 환경 등, 여러가지가 있다.
	
	다음은 \textbf{패키지} 이다. 패키지는 기본 제공되는  환경 외에 환경안의 스크립트가 어떠한 역할을 할지 알려주는 라이브러리와 같은 역할을 한다. 만약 여러분이 \TeX Live를 통하여 테글 설치 했다면 이 책에서 다루는 패키지는 전부 설치가 되었을 것이다.패키지는 프림블 \textbf{\textbackslash usepackage\{sample\}}\를 입력하여 사용 할 수 있다.
	\clearpage
	\subsection{Hello World!!}
	\LaTeX 를 사용하는 이유, 장점, 문서구조에 대하여 배우고 설치 까지 완료를 했으면 이젠 실제로 문서를 작성할 시간이 되었다. \TeX Editor을 키고 새 문서를 만들어 보자. 우리는 이젠 코딩의 관례에 따라 Hello World!!가 쓰인 문서를 작성해 볼 것이다.\newline
	
		\begin{tabularx}{\textwidth \onehalfspacing}{ |X|X| }		
			\hline
			\textbackslash documentclass\{article\}&Hello World!!\\
			\ \ \ \textbackslash begin\{document\}		& \\
			\ \ \ \ \ \ Hello World!!			& \\
			\ \ \ \textbackslash end\{document\}			& \\
			\hline
		\end{tabularx}
		\newline
		\newline
	{\bf 이 책에서의 예시는 따로 언급이 없다면 왼쪽에는 \LaTeX 코드를 오른쪽에는 컴파일 결과를 표시 하겠다.}\newline
	\underline{만약 {\bf 한국어}를 입력하고 싶다면 프림블에 `\textbackslash usepackage\{kotex\}'를 입력하고 사용하면 된다.}\newline\newline
	여러분은 이제 Hello World!!가 적힌 문서를 \LaTeX 를 통하여 만들 수 있게 되었다. 하지만 여러분은 이 Hello World!! 가 너무 평범해 보인다는 것에 동의를 할 것이다. 우리는 여러분의 기대를 저버리지 않기 위해 이제 Hello World!!가 다양한 글꼴을 가지게 할 것이다.
	\subsubsection{굵게, 기울임꼴, 밑줄}
	아래는 글자를 {\it italic}, {\bf bold}로 작성을 하거나 \underline{밑줄}치는 예시이다.
	\newline
	{\bf 이 예제부터는 따로 패키지를 사용하지 않는 경우 본문 안의 코드만 표기 할 것이다}\newline
	
	\begin{tabularx}{\textwidth \onehalfspacing}{ |X|X| }
		\hline
		\textbackslash textit\{Hello World!!\} & \textit{Hello World!!}\\
		\{\textbackslash it\ Hello World!! \} & {\it Hello World!!}\\
		\hline
		\textbackslash textbf\{Hello World!!\} & \textbf{Hello World!!}\\
		\{\textbackslash bf\ Hello World!! \} & {\bf Hello World!!}\\
		\hline
		\textbackslash underline\{Hello World!!\} & \underline{Hello World!!}\\
		\hline
	\end{tabularx}
	\newline
	\newline
	눈치 빠른 여러분들은 알아 차렸듯이 italic이나 bold 체는 사용을 할 수 있는 방법이 두 가지가 있다.여러분이 사용하기 편한대로 사용하면 된다.
	\clearpage
	\subsubsection{글꼴 크기}
	글자를 italic, bold 그리고 밑줄까지 쳤으면 이젠 문자의 크기를 조절 할 차례이다.\newline
	
	\begin{tabularx}{\textwidth \onehalfspacing}{|X|X|}
		\hline
		\textbackslash tiny Hello World!! & \tiny Hello World!!\\
		\textbackslash scriptsize Hello World!! & \scriptsize Hello World!!\\
		\textbackslash footnotesize Hello World!! & \footnotesize Hello World!!\\
		\textbackslash small Hello World!! & \small Hello World!!\\
		\textbackslash normalsize Hello World!! & \normalsize Hello World!!\\
		\textbackslash large Hello World!! & \large Hello World!!\\
		\textbackslash Large Hello World!! & \Large Hello World!!\\
		\textbackslash LARGE Hello World!! & \LARGE Hello World!!\\
		\textbackslash huge Hello World!! & \huge Hello World!!\\
		\textbackslash Huge Hello World!! & \Huge Hello World!!\\
		\hline
	\end{tabularx}
	\newline
	\newline
	\textbf{참고: 이 책은 12pt article문서 클래스를 사용하고 있어 \textbackslash huge와 \textbackslash Huge의 크기 차이가 없다.}
	
	위와 같은 방법을 사용하면 \textbackslash size 뒤에 있는 {\bf 모든 텍스트 크기가} 변하게 된다. 만약 {\large 이것처럼} 특정 부분에만 글자 크기를 적용 하고 싶다면 \newline
	
	\begin{tabularx}{\textwidth \doublespacing}{ |X|X| }
		\hline
		\{\textbackslash LARGE LARGE\} normal & {\LARGE LARGE} normal\\
		\hline
	\end{tabularx}
	\newline
	\newline
	와 같이 사용하면 된다.
	\subsubsection{특수 기호}
	이젠 특수부호를 입력하는 방법을 배울 시간이다. WYSIWYG인 워드 프로세서와 달리 \LaTeX 와 같은 WYSIWYM인 문서를 작성을 할때에는 많은 특수기호들이 명령어와 겹치기 때문에 일반적인 방법으로 특수 문자들을 입력을 해줄 수 없다. 몇가지 특수기호들의 예시를 들어 보겠다.\newline
	
	\begin{tabularx}{\textwidth \onehalfspacing}{|X|X|}
		\hline
		\textbackslash \%, \textbackslash \&, \textbackslash \_, \textbackslash \$, \textbackslash \#, \textbackslash \{, \textbackslash\} & \%, \&, \_, \$, \#,\{, \}\\
		\hline
	\end{tabularx}
	\clearpage
	\subsubsection{내장 폰트}
	\TeX 는 다음과 같은 내장 폰트를 제공한다.\newline
	
	\begin{tabularx}{\textwidth\onehalfspacing}{|X|X|}
		\hline
		\textbackslash emph\{Emphasis\}&\emph{Emphasis}\\
		\hline
		\textbackslash textrm\{Roman Font Family\}&\textrm{Roman Font Family}\\
		\hline
		\textbackslash textsf\{Sans Serif Font Family\}&\textsf{Sans Serif Font Family}\\
		\hline
		\textbackslash texttt\{Teletypefont Family\}&\texttt{Teletypefont Family}\\
		\hline
		\textbackslash textsc\{Small Capitals\}& \textsc{Small Capitals}\\
		\hline
	\end{tabularx}\newline\newline
	\TeX 는 학술 논문 작성을 위해 만들어진 언어 이므로 이 이외의 폰트를 사용 할 일이 거의 없을 것이다. 하지만 만약 다른 폰트를 사용하고 싶으면 구글에 검색을 해보기를 추천한다.
	\vfill 축하한다! 여러분은 이젠 기본적인 문장을 \LaTeX \로 작성할 수 있게 되었다.
	\clearpage
	
	\subsection{\LaTeX 단락 설정하기}
	\subsubsection{이 단원의 목표}
	이번 장에서 여러분은 다음 사항을 학습 할 것이다.
	\begin{itemize}
		\item 개행하기
		\item 좌측, 우측, 중간 정렬
		\item 아래 정렬
		\item 줄간격 조절
		\item 주석, 메모, 코멘트
		\item Section 나누기
	\end{itemize} 
	\subsubsection{개행하기}
	개행하는 방법에는 총 3가지가 있으며 2가지 결과가 나오게 된다.\newline
	
	\begin{tabularx}{\textwidth \onehalfspacing}{|X|X|}
		\hline
		Hello World!! & Hello World!!\\
		\ & \ \ \ \ \ \ Hello World!!\\
		Hello World!! & \\
		\hline
		Hello World!!\textbackslash newline Hello World!! & Hello World!!\\
		\ &Hello World!!\\
		\hline
		Hello World!! \textbackslash \textbackslash Hello World!! & Hello World!!\\
		\ & Hello World!!\\
		\hline
	\end{tabularx}
	\newline
	\newline
	위에서 볼 수 있듯이 두 번 개행할 경우 새로운 문단으로 시작하게 되며 다른 두 방법은 줄만 바뀌게 된다.
	\subsubsection{좌측, 우측, 중간 정렬}
	좌측, 우측, 중간 정렬은 적용할 글자가 적을 경우에는 다음과 같은 방법을 사용하면 된다.\newline
	
	\begin{tabularx}{\textwidth \onehalfspacing}{|X|X|}
		\hline
		\{\textbackslash flushleft Hello World!! \} & Hello World!!\\
		\hline
		\{\textbackslash flushright Hello World!! \} & \multicolumn{1}{r|}{Hello World!!}\\
		\hline
		\{\textbackslash centering Hello World!! \} & \multicolumn{1}{c|}{Hello World!!}\\
		\hline
	\end{tabularx}
	\clearpage
	적용할 글자가 한 문단이거나 많을 경우에는 다음과 같은 방법을 사용하는 것을 추천한다. \newline
	
	\begin{tabularx}{\textwidth \onehalfspacing}{|X|X|}
		\hline
		\textbackslash begin\{flushleft\} & Hello World!!\\
		\ \ \ \ \ \ Hello World!! & \\
		\textbackslash end\{flushleft\} & \\
		\hline
		\textbackslash begin\{flushright\} & \multicolumn{1}{r|}{Hello World!!}\\
		\ \ \ \ \ \ Hello World!! & \\
		\textbackslash end\{flushright\} & \\
		\hline
		\textbackslash begin\{center\} & \multicolumn{1}{c|}{Hello World!!}\\
		\ \ \ \ \ \ Hello World!! & \\
		\textbackslash end\{center\} & \\
		\hline
	\end{tabularx}
	\subsubsection{아래정렬}
	이 책의 표지와 같이 글을 페이지 아래에 적어야 할때도 있다. 그럴 경우 아래와 같은 방법을 사용하면 된다.\newline
	
	\begin{tabularx}{\textwidth \onehalfspacing}{|X|X|}
		\hline
		\textbackslash vfill Hello World!! & \\
		 & \\
		 & Hello World!!\\
		\hline
	\end{tabularx}
	\subsubsection{줄간격 조절하기}
	우리는 가독성을 높이기 위해서 줄간격을 조절 할 때가 있다. 여러분은 크게 2가지 방식으로 줄간격을 조절 할 수가 있다.\newline
	
	\begin{tabularx}{\textwidth}{|X|X|}
		\hline
		Contents... &The quick brown fox jumps over the lazy dog\\
		\hline
		\textbackslash usepackage\{setspace\} 
		
		//...
		
		\textbackslash onehalfspacing Contents... & \onehalfspacing The quick brown fox jumps over the lazy dog\\
		\hline
		\textbackslash usepackage\{setspace\} 
		
		//...
		
		\textbackslash  doublespacing Contents... & \doublespacing The quick brown fox jumps over the lazy dog\\
		\hline
	\end{tabularx}
	\newline
	\newline
	위와 같이 사용 할 경우 \textbackslash \_\_\_\_spacing 뒤의 모든 텍스트들에 적용이 되므로 예를 들어 한 문단에만 적용을 하고 싶으면 다음과 같이 사용하면 된다.\newline
	
	\begin{tabularx}{\textwidth}{|X|X|}
		\hline
		Contents... & The quick brown fox jumps over the lazy dog\\
		\hline
		\textbackslash usepackage\{setspace\} 
		
		//...
		
		\textbackslash \{spacing\}\{1.5\}
		
		Contents...
		
		\textbackslash \{spacing\} & \begin{spacing}{1.5} 
			The quick brown fox jumps over the lazy dog
		\end{spacing}\\
		\hline
	\end{tabularx}
	\subsubsection{주석, 메모, 코멘트}
	코드를 작성할 때 여러분은 주석 처리 하거나, 또는 문서 작성을 할 때 메모나 코멘트를 해야 하는 경우가 있었을 것이다.\newline
	주석 처리를 하기 위해서는 다음과 같은 방법을 사용한다.\newline
	
	\begin{tabularx}{\textwidth \onehalfspacing}{|X|X|}
		\hline
		visible dog & visible dog\\
		\% invisible cat & \\
		\hline
	\end{tabularx}
	\newline
	\newline
	많은 글을 주석 처리를 해야 할 경우에는 다음과 같은 방법을 사용하기를 권장한다.\newline
	
	\begin{tabularx}{\textwidth \onehalfspacing}{|X|X|}
		\hline
		\textbackslash usepackage\{verbatim\} & visible dog\\
		//...& \\
		visible dog & \\
		\textbackslash begin\{comment\} & \\
		\ \ \ \ \ \ invisible cat & \\
		\textbackslash end\{comment\} & \\
		\hline
	\end{tabularx}
	\newline
	\newline
	그 외에도 여러가지 주석처리나 메모기능이 있으며 만약 다른 방법을 사용하고 싶다면 인터넷에서 찾아보기 바란다.
	\subsubsection{Section 나누기}
	논문이나 보고서를 작성을 할 때 이 책 처럼 문서를 장, 단원, 등으로 나누어야 할 때가 있다. \LaTeX 를 사용하면 손 쉽게 Section 나누기를 할 수가 있다.\LaTeX 특성상 표 안에 예시 결과를 넣을 수 없기 떄문에 2장의 Section을 예시로 하겠다.\newline
	
	\begin{tabularx}{\textwidth \onehalfspacing}{|X|}
		\hline
		\textbackslash section\{\LaTeX 와 친해지기 \}\\
		\textbackslash subsection\{이번장의 목표\}\\
		\textbackslash subsection\{\LaTeX 의 문서구조 \}\\
		\textbackslash subsubsection\{Preamble\}\\
		\textbackslash subsubsection\{Body\}\\
		\hline
	\end{tabularx}
	\newline
	\newline
	와 같이 사용하면 2장의 2.2까지 Section으로 나눌 수가 있다. 내용은 Section 사이사이에 들어간다.
	\clearpage
	
	\section{수식에 특화된 \LaTeX}
	\subsection{이번 장의 목표}
	\begin{itemize}
		\item 수식 입력하기
		\item 기호 입력하기
		\item 연산자 입력하기
		\item 액센트 입력
	\end{itemize}
	\subsection{수식 입력하기}
	\LaTeX 로 논문이나 보고서를 작성 할 때 수식을 넣어야 하는 경우가 많을 것이다. 수식을 입력 할때 다음과 같은 환경을 만들어 주어야 한다.\newline
	
	\begin{tabularx}{\textwidth \onehalfspacing}{|X|X|}
		\hline
		Equation: \textbackslash (1+b=3\textbackslash ) & Equation: \(1+b=3\)\\
		\hline
		Equation: \textbackslash[1+b=3\textbackslash] & Equation: \[1+b=3\]\\
		\hline
		Equation: \$1+b=3\$ & Equation: $1+b=3$\\
		\hline
		Equation: \$\$1+b=3\$\$ & Equation: $$1+b=3$$\\
		\hline
		Equation:
		\textbackslash begin\{math\}
		
		1+b=3
		
		\textbackslash end\{math\} & 
		Equation:
		\begin{math}
		1+b=3
		\end{math}\\
		\hline
		Equation:
		\textbackslash begin\{displaymath\}
		
		1+b=3
		
		\textbackslash end\{displaymath\} & 
		Equation:
		\begin{displaymath}
		1+b=3
		\end{displaymath}\\
		\hline
	\end{tabularx}
	\newline
	\newline
	홀수번째와 짝수번째의 차이는 수식을 텍스트와 같이 있느냐 아니면 따로 디스플레이 하느냐에 있다. \newline
	{\bf 주의: \$\$...\$\$는 AMS-\LaTeX 같은 매크로와 충돌이 생겨 문제가 생길 수 있으므로 사용을 지양해야 한다.}\clearpage
	\subsection{기호 입력하기}
	수학에서는 여러가지 기호가 사용이 된다. 여러분은 그 여러가지 기호를 수식에도 입력을 해야 하기 때문에 이 단원에는 여러가지 기호의 예시를 보여 주도록 하겠다.\newline\newline
	처음으로 바로 사용할 수 있는 기호목록이다.\newline
	{\bf 이 장의 예제는 이제부터 예시는 수식 환경 안에 있는 내용만 표기 하겠다.}\newline
	
	\begin{tabularx}{\textwidth \onehalfspacing}{|X|}
		\hline
		\(+ - / = ! ( ) [ ] < > | ' :\)\\
		\hline
	\end{tabularx}
	\newline
	\newline
	그리스 문자들은 다음과 같이 사용하면 된다.\newline
	
	\begin{tabularx}{\textwidth \onehalfspacing}{|X|X|}
		\hline
		\textbackslash alpha, \textbackslash beta, \textbackslash gamma, \textbackslash pi, \textbackslash phi, \textbackslash varphi & \(\alpha \beta \gamma \pi \phi \varphi\)\\
		\hline
	\end{tabularx}
	\newline\newline
	다음은 삼각함수 사용의 예시이다.\newline
	
	\begin{tabularx}{\textwidth \onehalfspacing}{|X|X||X|X||X|X||X|X|}
		\hline
		수식&코드&수식&코드&수식&코드&수식&코드\\
		\hline
		\hline
		\(\sin\)&\textbackslash sin &
		\(\cos\)&\textbackslash cos &
		\(\tan\)&\textbackslash tan &
		\(\cot\)&\textbackslash cot \\
		\hline
		\(\arcsin\)&\textbackslash arcsin &
		\(\arccos\)&\textbackslash arccos &
		\(\arctan\)&\textbackslash arctan &
		&\\
		\hline
		\(\sinh\)&\textbackslash sinh &
		\(\cosh\)&\textbackslash cosh &
		\(\tanh\)&\textbackslash tanh &
		\(\coth\)&\textbackslash coth \\
		\hline
		\(\sec\)&\textbackslash sec &
		\(\csc\)&\textbackslash csc &
		 & & & \\
		 \hline
	\end{tabularx}
	\newline
	\newline
	다음은 논리 기호 입력 방법이다.\newline
	
	\begin{tabularx}{\textwidth \onehalfspacing}{|l|X||l|X||l|X||l|X|}
		\hline
		기호&코드&기호&코드&기호&코드&기호&코드\\
		\hline
		\hline
		\(\exists\)&\textbackslash exists&
		\(\nexists\)&\textbackslash nexists&
		\(\forall\)&\textbackslash forall&
		&\\
		\hline
		\(\subset\)&\textbackslash subset&
		\(\supset\)&\textbackslash supset&
		\(\in\)&\textbackslash in&
		\(\ni\)&\textbackslash ni\\
		\hline
		\(\notin\)&\textbackslash notin &
		\(\emptyset \)&\textbackslash emptyset &
		\(\top\)&\textbackslash top &
		\(\bot\)&\textbackslash bot \\
		\hline
		\(\rightarrow\)&\textbackslash rightarrow &
		\(\leftarrow\)&\textbackslash leftarrow &
		\(\Rightarrow\)&\textbackslash Rightarrow &
		\(\Leftarrow\)&\textbackslash Leftarrow \\
		\hline
		\(\leftrightarrow\)&\small\textbackslash leftrightarrow &
		\(\Leftrightarrow \)&\small\textbackslash Leftrightarrow &
		\(\land\)&\textbackslash land &
		\(\lor\)&\textbackslash lor \\
		\hline
	\end{tabularx}
	\clearpage
	다음은 관계기호 입력 방법이다.\newline
	
	\begin{tabularx}{\textwidth \onehalfspacing}{|l|X||l|X||l|X||l|X|}
		\hline
		기호&코드&기호&코드&기호&코드&기호&코드\\
		\hline
		\hline
		\(\leq\)&\textbackslash leq &
		\(\geq\)&\textbackslash geq &
		\(\ll\)&\textbackslash ll &
		\(\gg\)&\textbackslash gg \\
		\hline
		\(\subset\)&\textbackslash subset &
		\(\supset\)&\textbackslash supset &
		\(\subseteq\)&\textbackslash subseteq &
		\(\supseteq\)&\textbackslash supseteq \\
		\hline
		\(\nsubseteq\)&\textbackslash nsubseteq &
		\(\nsupseteq\)&\textbackslash nsupseteq &
		\(\parallel\)&\textbackslash parallel &
		\(\nparallel\)&\textbackslash nparallel \\
		\hline
		\(\prec\)&\textbackslash prec &
		\(\succ\)&\textbackslash succ &
		\(\preceq\)&\textbackslash preceq &
		\(\succeq\)&\textbackslash succeq \\
		\hline
		\(\doteq\)&\textbackslash doteq &
		\(\equiv\)&\textbackslash equiv &
		\(\approx\)&\textbackslash approx &
		\(\cong\)&\textbackslash cong \\
		\hline
		\(\sim\)&\textbackslash sim &
		\(\simeq\)&\textbackslash simeq &
		\(\neq\)&\textbackslash neq &
		\(\propto\)&\textbackslash propto \\
		\hline
	\end{tabularx}
	\newline\newline
	여기에 없는 다른 기호들은 인터넷에서 찾아보거나 \TeX studio를 사용할 경우 원쪽 바에 수식 아카이브가 있으므로 여러분은 유용하게 사용하기만 하면 된다.
	\subsection{연산자 입력하기}
	여러분은 \LaTeX 에서 여러가지 수학 기호를 배웠다. 이젠 배운 수학기호들을 사용 할 때가 왔다. 다음은 연산자를 이용한 수식입력의 예시이다.\newline
	
	\begin{tabularx}{\textwidth \large \onehalfspacing}{|X|X|}
		\hline
		k\_1+k\textasciicircum2=3 & \(k_1+k^2=3\)\\
		\hline
		\textbackslash lim\_\{x \textbackslash to\ \textbackslash infty\}x=\textbackslash infty&\(\lim_{x \to \infty}x=\infty \)\\
		\hline
		\textbackslash exp(a)=1&\(\exp(a)=1\)\\
		\hline
		\textbackslash cos(3\textbackslash theta)=\textbackslash cos\textasciicircum3 \textbackslash theta & \(\cos(3\theta)=\cos^3\theta\)\\
		\hline
		x \textbackslash equiv 3 \textbackslash pmod\{4\}&\(x \equiv 3 \pmod{4}\)\\
		\hline
		\textbackslash frac\{a\}\{4\} & \(\frac{a}{4} \)\\
		\hline
		\textbackslash displaystyle \textbackslash sum\_\{k=0\}\textasciicircum\{10\}x&\(\displaystyle \sum_{k=0}^{10}x \)\\
		\hline
		\textbackslash int\_0\textasciicircum \textbackslash infty x\textbackslash ,\textbackslash mathrm\{d\}x& \(\int_{0}^{\infty}x\,\mathrm{d}x \)\\
		\hline
		\textbackslash frac\{\textbackslash mathrm d\}\{\textbackslash mathrm d x\} \textbackslash big( k f(x) \textbackslash big)&\(\frac{\mathrm{d}}{\mathrm d x}\big( k g(x) \big) \)\\
		\hline
		x \textbackslash in [-2,1]&\(x\in [-2,1] \)\\
		\hline
		\small\textbackslash begin \{bmatrix\}\newline
		1\&0\&1 \textbackslash\textbackslash
		0\&1\&0 \newline
		\textbackslash end\{bmatrix\}&\ \small\newline
		\(\begin{bmatrix}
		1&0&1\\0&1&0
		\end{bmatrix} \)\\
		\hline
	\end{tabularx}
	\newline\newline
	여기에 없는 다른 수식/연산자 들은 인터넷에서 찾아보자.
	\subsection{엑센트 입력하기}
	수식을 입력할 때에 벡터와 같이 엑센트를 사용 하는 경우가 적지 않게 있다. 엑센트를 사용하는 예시는 다음과 같다.\newline
	
	\begin{tabularx}{\textwidth \onehalfspacing}{|l|X||l|X|}
		\hline
		\(a'\) & a' & \(a''\) & a''\\
		\hline
		\(\hat{a} \) & \textbackslash hat\{a\} & \(\bar{a} \) & \textbackslash bar \{a\}\\
		\hline
		\(\dot{a}\) & \textbackslash dot\{a\} & \(\ddot{a}\) &\textbackslash ddot\{a\}\\
		\hline
		\(\overrightarrow{AB}\) & \textbackslash overrightarrow\{AB\} & \(\overleftarrow{AB}\) & \textbackslash overleftarrow\{AB\}\\
		\hline
		\(\overline{abc}\) & \textbackslash overline\{abc\}& \(\vec{v}\) & \textbackslash vec\{v\} \\
		\hline
	\end{tabularx} 
	\clearpage
	\section{여러가지 요소삽입}
	\subsection{이번장의 목표}
	\begin{itemize}
		\item 목차 삽입하기
		\item 각주 삽입하기
		\item 아이템 열거 삽입하기
		\item 이미지 삽입하기
		\item 표 삽입하기
		\item 그래프 삽입하기
	\end{itemize}
	\subsection{목차 삽입하기}
	책이나 논문은 많은 section으로 나위어 지는 경우가 많으며 이 section의 페이지나 대략적인 내용을 알려주기 위하여 목차를 넣는 경우가 많다. \LaTeX 에서는 간단하게 한 스크립트만 입력을 해주면 목차를 작성해 주게 된다.\newline
	
	\begin{tabularx}{\textwidth \onehalfspacing}{|X|}
		\hline
		\textbackslash tableofcontents\\
		\hline
	\end{tabularx}
	\newline\newline
	\textbf{주의: }목차는 두번 컴파일 해 주어야 업데이트가 된다.\\
	다음은 그림, 표 목록과 켭션을 출력하기 위해서는 다음 스크립트를 입려해 주면 된다.\newline
	
	\begin{tabularx}{\textwidth \onehalfspacing}{|X|}
		\hline
		\textbackslash listoffigures\\
		\textbackslash listoftables\\
		\hline
	\end{tabularx}
	\subsection{각주 삽입하기}
	각주 입력하기는 매우 쉽다.\\
	
	\begin{tabularx}{\textwidth \onehalfspacing}{|X|X|}
		\hline
		Where is footnote?\textbackslash footnote\{Here it is!\}&Where is footnote? \footnote{Here it is!}\\
		\hline
	\end{tabularx}
	\newline\newline
	각주 입력하기는 매우 쉽다. 하지만 여러분은 각주의 위치에 주의해야 한다. 각주는 \textbf{마지막 텍스트} 다음 줄에 위치하게 된다. 그럼으로 만약 각주가 페이지 가장 밑에 있기를 원한다면 그 페이지 마지막 문장 다음에 \textbackslash clearpage를 해주는 것이 좋다.\clearpage
	\subsection{아이템 열거 삽입하기}
	매장 시작마다 여러분은 단원을 나열한 리스트를 볼 수가 있었을 것이다.이와 같은 아이템 리스트 들은 다음과 같이 서식 할 수가 있다.\newline
	
	\begin{tabularx}{\textwidth \onehalfspacing}{|X|X|}
		\hline
		\textbackslash begin\{itemize\}
		
		\ \ \ \ \ \ \textbackslash item Chocola
		
		\ \ \ \ \ \ \textbackslash item Vanilla
		
		\ \ \ \ \ \ \textbackslash item Maple
		
		\textbackslash end\{itemize\}
		&
		\begin{itemize}
			\item Chocola
			\item Vanilla
			\item Maple
		\end{itemize}
		\\
		\hline
	\end{tabularx}
	\newline\newline
	만약 번호가 있는 열거를 원한다면 enumerate환경을 사용하면 된다.\newline
	
	\begin{tabularx}{\textwidth \onehalfspacing}{|X|X|}
		\hline
		\textbackslash begin\{enumerate\}
		
		\ \ \ \ \ \ \textbackslash item Chocola
		
		\ \ \ \ \ \ \textbackslash item Vanilla
		
		\ \ \ \ \ \ \textbackslash item Maple
		
		\textbackslash end\{enumerate\}
		&
		\begin{enumerate}
			\item Chocola
			\item Vanilla
			\item Maple
		\end{enumerate}
		\\
		\hline
	\end{tabularx}
	\clearpage
	\subsection{이미지 삽입하기}
	이미지는 다음과 같이 삽입하면 된다.\newline
	\textbf{표 안에는 이미지를 추가 할 수 없으므로 이미지는 표 바로 아래 추가한다.}\newline
	
	\begin{tabularx}{\textwidth \onehalfspacing}{|X|}
		\hline
		\textbackslash begin\{figure\}[h]
		
		\ \ \ \ \ \ \textbackslash centering
		
		\ \ \ \ \ \ \textbackslash includegraphics[scale=1]\{image\_name.png\}
		
		\textbackslash end\{figure\}
		\\
		\hline
	\end{tabularx}
	\begin{figure}[h]
		\centering
		\includegraphics[scale=1]{T6.png}
	\end{figure}
	\newline 이미지에 캡션을 달기 위해서는 간단히 스크립트 한 줄만 더 적어주면 된다.\newline
	
	\begin{tabularx}{\textwidth \onehalfspacing}{|X|}
		\hline
		\textbackslash begin\{figure\}[h]
		
		\ \ \ \ \ \ \textbackslash centering
		
		\ \ \ \ \ \ \textbackslash includegraphics[scale=1]\{image\_name.png\}
		
		\ \ \ \ \ \ \textbackslash caption\{Example Image duke\}
		
		\textbackslash end\{figure\}\\
		\hline
	\end{tabularx}
	\begin{figure}[h]
		\centering
		\includegraphics[scale=1]{T6.png}
		\caption{Example Image duke}
	\end{figure}
	\clearpage 위 두 예시를 보고 여러분은 왜 \textbf{[h]}가 무슨 기능을 하는지 궁금했을 것이다. 대괄호(\textbf{[ ]})안에 무엇이 들어가느냐에 따라서 이미지가 삽입되는 위치가 변하게 된다.자세한 내용은 아래를 참조 바란다.\newline
	
	\begin{tabularx}{\textwidth \onehalfspacing}{|l|X|}
		\hline
		h&(\textbf{h}ere) 개체를 코드 위치에 삽입\\
		\hline
		t&(\textbf{t}op) 개체를 페이지 맨 위에 삽입\\
		\hline
		b&(\textbf{b}ottom) 개체를 페이지 맨 아래에 삽입\\
		\hline
		htbp& 개채를 코드 위치에 우선적으로 삽입하고 페이지 경계를 넘어가거나 할 경우에 t,b,p와 같은 차선책을 사용하여 삽입\\
		\hline
	\end{tabularx}
	\newline\newline
	두단으로 나뉘어진 양식을 채택한 문서나 학술지 같은 경우 이미지를 중감에 삽입하기 보다 위, 안 될 경우 아래에 삽입하는 것을 추천한다.
	\newline\newline
	삽입할 사진의 유형에 따라 적합한 이미지 양식은 다음과 같다.
	\begin{itemize}
		\item 사진: jpg 또는 jpeg
		\item 그래프:pdf 또는 eps
		\item 기하적 그림: png 또는 벡터 이미지
		\item 일러스트: pdf (PowerPoint에서 `pdf로 내보내기')
	\end{itemize}
	\clearpage
	\subsection{표 삽입하기}
	여러분은 이 책을 보면서 수 많은 표를 보았을 것이다. 이젠 여러분이 직접 표를 만들 수 있게 표 만드는 방법을 가르쳐 주겠다.맨 처음으로 표 제작은 \textbf{tablar}환경을 사용한다.\newline
	\begin{center}
		\onehalfspacing
		\begin{tabular}{|l|}
			
			\hline
			\textbackslash begin\{tablar\}\{\textbar l\textbar\textbar c\textbar r\textbar\}\\		
			\ \ \ \ \ \ \textbackslash hline\\		
			\ \ \ \ \ \ name\&gender\&offshore cash\textbackslash\textbackslash\\
			\ \ \ \ \ \ \textbackslash hline\\
			\ \ \ \ \ \ \textbackslash hline\\
			\ \ \ \ \ \ Trevor\&male\&200,000,000\$\textbackslash\textbackslash\\
			\ \ \ \ \ \ \textbackslash hline\\
			\ \ \ \ \ \ Michael\&male\&17,000,000\$\textbackslash\textbackslash\\
			\ \ \ \ \ \ \textbackslash hline\\
			\textbackslash end\{tabular\}\\
			\hline
		\end{tabular}
		\ \ \ \ \ \ \begin{tabular}{|l||c|r|}
			\hline
			name&gender&offshore cash\\
			\hline
			\hline
			Trevor&male&200,000,000\$\\
			\hline
			Michael&male&17,000,000\$\\
			\hline
		\end{tabular}
	\end{center}
	이젠 위 스크립트를 하나씩 알아볼 시간이다.\textbf{\{ \textbar\ ㅣ \textbar\ c \textbar\ r \textbar\}}\는 각 열을 촤즉, 중간, 우측 정렬 시켜주며, 사이의 \textbf{``\textbar''}\은 수직선을 나타낸며 여러번 사용이 가능하고, \textbf{\textbackslash hline}\은 수평선을 나타내며 역시 여러번 사용 가능하다. 열 구분 기호로는 \textbf{\&}\가 사용이 되며, 행 구분 기호로는 \textbf{\textbackslash\textbackslash}\가 사용이 된다.\newline\newline
	다음은 여러가지 그래프를 그리는 예시이다.\newline
	\textbf{이번 표 예시부터는 tabular화경안에 있는 스크립트만 서술하겠다.}\newline
	\begin{center}
		\onehalfspacing
		\begin{tabular}{|l|}
			\hline
			\textbackslash hline\\
			\textbackslash diagbox\{down\}\{up\}\textbackslash\textbackslash\\
			\textbackslash hline\\
			\hline
		\end{tabular}
		\ \ \ \ \ \ 
		\begin{tabular}{|l|}
			\hline
			\diagbox{down}{up}\\
			\hline
		\end{tabular}
	\end{center}
	
	\begin{center}
		\onehalfspacing
		\begin{tabular}{|l|}
			\hline
			\textbackslash hline\\
			short\&short\&short\textbackslash\textbackslash\\
			\textbackslash hline\\
			\textbackslash multicolumn\{3\}\{\textbar c\textbar\}\{long\}\textbackslash\textbackslash\\
			\textbackslash hline\\
			\hline
		\end{tabular}
		\ \ \ \ \ \ 
		\begin{tabular}{|c|c|c|}
			\hline
			short&short&short\\
			\hline
			\multicolumn{3}{|c|}{long}\\
			\hline
		\end{tabular}
	\end{center}
	\clearpage
	\begin{center}
		\onehalfspacing
		\begin{tabular}{|l|}
			\hline
			\textbackslash hline\\
			short\&short\&short\textbackslash\textbackslash\\
			\textbackslash hline\\
			short\&\textbackslash multicolumn\{2\}\{c\textbar\}\{long\}\textbackslash\textbackslash\\
			\textbackslash hline\\
			\hline
		\end{tabular}
		\ \ \ \ \ \ 
		\begin{tabular}{|c|c|c|}
			\hline
			short&short&short\\
			\hline
			short&\multicolumn{2}{c|}{long}\\
			\hline
		\end{tabular}
		\newline
	\end{center}
	\textbf{주의: 위 예제와 같은 경우에는 ``c\textbar'' 대신 ``\textbar c\textbar''\를 사용 할 경우 수직선이 중복되어 short와 long사이의 수직선만 두껍게 표기가 된다.}\newline
	\begin{center}
		\onehalfspacing
		\begin{tabular}{|l|}
			\hline
			normal\\
			\textbackslash hline
			\textbackslash hline\\
			\textbackslash multicolumn\{1\}\{\textbar l \textbar\}\{left\}\textbackslash\textbackslash\\
			\textbackslash hline\\
			\textbackslash multicolumn\{1\}\{\textbar r \textbar\}\{right\}\textbackslash\textbackslash\\
			\textbackslash hline\\
			\textbackslash multicolumn\{1\}\{\textbar c \textbar\}\{center\}\textbackslash\textbackslash\\
			\textbackslash hline\\
			\hline
		\end{tabular}
		\ \ \ \ \ \ 
		\begin{tabular}{|l|}
			\hline
			normal\\
			\hline
			\multicolumn{1}{|l|}{left}\\
			\hline
			\multicolumn{1}{|r|}{right}\\
			\hline
			\multicolumn{1}{|c|}{center}\\
			\hline
		\end{tabular}
	\end{center}\clearpage
	\subsection{그래프 삽입}
	\subsubsection{tikz}
	tikz를 사용하여 \LaTeX 에서 그래프 및 도형을 그릴 수가 있다. tikz를 사용하기 위해서는 tikz 패키지를 추가 해야 한다.\newline tikz를 사용 할 때 다음 사항을 주의 해야 한다.
	\begin{itemize}
		\item 데카르트 좌표계나 극 좌표계가 사용이 된다.
		\item 오른쪽으로 x좌표가 증가하고 위쪽으로 y좌표가 증가한다.
		\item 중심이 (0,0)이 아니라 상황에 따라 알맞은 위치로 맞춰진다.
	\end{itemize}
	tikz사용을 위해서는 다음과 같은 환경을 만드면 된다.\newline
	
	\begin{tabularx}{\textwidth \onehalfspacing}{|X|}
		\hline
		\textbackslash usepackage\{tikz\}\\
		//...\\
		\textbackslash begin\{tikzpicture\}\\
		\ \ \ \ \ \ \%scripts\\
		\textbackslash end\{tikzpicture\}\\
		\hline
	\end{tabularx}\newline\newline
	또는 \newline
	
	\begin{tabularx}{\textwidth \onehalfspacing}{|X|}
		\hline
		\textbackslash usepackage\{tikz\}\\
		//...\\
		\textbackslash tikz\{\%scripts\}\\
		\hline
	\end{tabularx}\newline\newline
	으로 사용해도 된다.\clearpage
	\subsubsection{점}
	tikz로 점을 입력할 때는 점은 나오지 않고 텍스트만 나오게 된다.\newline
	\textbf{여러분도 이젠 예상할 수 있듯이 다음 예제 부터는 tikz환경 안에 있는 내용만 서술 할 것이다.}\newline
	
	\begin{tabularx}{\textwidth \onehalfspacing}{|X|X|}
		\hline
		\textbackslash node (a) at (0,0)\{a\};
		
		\textbackslash node (b) at (-2,2)\{b\};
		&\tikz{\node (a) at (0,0){a};\node (b) at (-2,2){b}}\\
		\hline
		\textbackslash path (0,0) node(x) \{x\} (2,2) node(y) \{y\}
		&\tikz{\path (0,0) node(x) {x} (2,2) node(y) {y}}\\
		\hline
	\end{tabularx}
	\subsubsection{선}
	선분을 그리는 방법은 다음과 같다.\newline
	
	\begin{tabularx}{\textwidth \onehalfspacing}{|X|X|}
		\hline
		\textbackslash draw[option] (0,0) - - (1,1) - - (2,0);
		&\tikz{\draw (0,0)--(1,1)--(2,0)}\\
		\hline
	\end{tabularx}\newline\newline
	두 점을 직선이 아닌 택시 거리로 연결 할 수도 있다.\newline
	
	\begin{tabularx}{\textwidth\onehalfspacing}{|X|X|}
		\hline
		\textbackslash draw (0,0) \textbar- (1,1);
		&\tikz{\draw (0,0) |-(1,1);}\\
		\hline
		\textbackslash draw (0,0) -\textbar (1,1);
		&\tikz{\draw (0,0) -| (1,1);}\\
		\hline
	\end{tabularx}\newline\newline
	다음 같은 방법으로 베지 곡선도 만들 수 있다.\newline
	
	\begin{tabularx}{\textwidth\onehalfspacing}{|X|X|}
		\hline
		\textbackslash draw (0,0) .. controls (1,1) .. (2,0);
		&\tikz{\draw (0,0) .. controls (1,1).. (2,0);}\\
		\hline
		\textbackslash draw (0,0) .. controls (1,1)
		
		 and (2,1) .. (3,0);
		&\tikz{\draw (0,0) .. controls (1,1)and(2,1).. (3,0);}\\
		\hline
	\end{tabularx} \clearpage
	화살표도 간단하게 만들 수 있다.\newline
	
	\begin{tabularx}{\textwidth\onehalfspacing}{|X|X|}
		\hline
		\textbackslash draw [-\textgreater] (0,0) - - (1,1) - -(3,0);&
		\tikz{\draw[->](0,0) -- (1,1)--(3,0);}\\
		\hline
		\textbackslash draw [\textless-\textgreater] (0,0) .. controls(1,1) ..(3,0);&
		\tikz{\draw[<->](0,0) .. controls(1,1)..(3,0);}\\
		\hline
	\end{tabularx}\newline\newline
	굵거나 점선으로 되어 있는 선분도 만들 수 있다.\newline
	
	\begin{tabularx}{\textwidth \onehalfspacing}{|X|X|}
		\hline
		\textbackslash draw[thick] (0,0) .. controls (1,1) .. (2,0);
		&\tikz{\draw[thick] (0,0) .. controls (1,1).. (2,0);}\\
		\hline
		\textbackslash draw[dashed] (0,0) .. controls (1,1) .. (2,0);
		&\tikz{\draw[dashed] (0,0) .. controls (1,1).. (2,0);}\\
		\hline
	\end{tabularx}
	\subsubsection{도형}
	tikz를 사용하여 직선 이외에 다양한 도형을 그릴 수 있다.이 때 \textbf{cycle}\을 사용하면 첫번째 점으로 이여진다.\newline
	
	\begin{tabularx}{\textwidth\onehalfspacing}{|X|X|}
		\hline
		\textbackslash draw[thick] (0,0) - - (1,1) - - (2,0)- -cycle;
		&\tikz{\draw[thick](0,0)--(1,1)--(2,0)--cycle;}\\
		\hline
		\textbackslash filldraw[blue] (0,0) - - (1,1) - - (2,0)- -cycle;
		&\tikz{\filldraw[blue](0,0)--(1,1)--(2,0)--cycle;}\\
		\hline
		\textbackslash draw(0,0) circle[radius=5mm];
		&\tikz{\draw (0,0)circle[radius=5mm];}\\
		\hline
		\textbackslash draw(0,0)&\\
		circle[x radius=10mm, y radius=5mm];
		&\tikz{\draw (0,0)circle[x radius=10mm, y radius=5mm];}\\
		\hline
		\textbackslash draw (0 ,0) arc (150:30:5mm);
		&\tikz{\draw (0,0) arc (150:30:5 mm );}\\
		\hline
		\textbackslash filldraw(0,0) - - (12 mm,0 mm) &\\arc(0:30:12 mm) - - cycle;
		&\tikz{\filldraw (0 ,0) -- (12 mm ,0 mm) arc (0:30:12 mm) -- cycle ;}\\
		\hline
	\end{tabularx}
	\clearpage
	\subsubsection{함수}
	좌표계를 그리는 방법은 다음과 같다.\newline
	
	\begin{tabularx}{\textwidth\onehalfspacing}{|X|c|}
		\hline
		\textbackslash draw[step =5mm, gray, thin](-1.2, -1.2) grid(1.2, 1.2);
		
		\textbackslash draw[-\textgreater, thick ](-1.25, 0)- -(1.25, 0);
		
		\textbackslash draw[-\textgreater, thick ](0, -1.25)- -(0, 1.25);
		&\tikz{\draw[step=5mm,gray,thin](-1.2 ,-1.2) grid(1.2,1.2);
			\draw[->, thick ]( -1.25 ,0) -- (1.25 ,0);
			\draw[->, thick ](0 , -1.25) -- (0 ,1.25);}\\
		\hline
	\end{tabularx}\newline\newline
	함수 그래프를 입력하기 위해서는 다음과 같이 정의역을 정해주고 사용하면 된다. \newline
	
	\begin{tabularx}{\textwidth\onehalfspacing}{|X|c|}
		\hline
		\textbackslash draw[step=5mm, gray, thin ](-1.2, -1.2) grid(1.2, 1.2);
		
		\textbackslash draw[domain=-1:1, samples=50] plot(\textbackslash x, {sin(pi*\textbackslash x r )});
		&\tikz{\draw[step=5mm, gray, thin](-1.2 ,-1.2) grid (1.2, 1.2);
			\draw[domain=-1:1, samples=50] plot(\x ,{ sin(pi*\x r )});}\\
		\hline
	\end{tabularx}
	\newline\newline
	여기서 ``pi*\textbackslash x''뒤의 `r'은 radian을 의미한다.
	\clearpage
	\section{Advanced Mathematics}
	\label{se:eq_label}
	\subsection{수식 라벨링}
	수학 논문의 경우 많은 수식이 사용되며 이것을 모두 번호를 매기고 참조를 하는데에 애를 먹을 수 있다. 심지어 중간에 수식을 첨부할 경우 뒤의 넘버링을 모두 바꾸어야 하고 상호 참조한 부분 까지 모두 일일이 찾아서 바꿔줘야 할 수도 있다. 하지만 \LaTeX 의 라벨링 기능을 사용하면 자동으로 상호참조 된 부분을 바꿔준다.
	\subsubsection{수식 라벨링}
	\textbf{equation}환경을 사용하면 자동으로 라벨링\를 해준다.\newline
	
	\begin{tabularx}{\textwidth\onehalfspacing}{|X|X|}
		\hline
		equation:
		
		\textbackslash begin\{equation\}
		
		\ \ \ \ \ \ f(x) = (x+1)(x+2)
		
		\textbackslash end\{equation\}
		&equation:
		\begin{equation}
			f(x) = (x+1)(x+2)
		\end{equation}\\
		\hline
	\end{tabularx}
	\newline\newline
	\textbf{amsmath}패키지를 사용하면 단원별로 라벨링이 가능하다. \newline
	
	\begin{tabularx}{\textwidth\onehalfspacing}{|X|}
		\hline
		\textbackslash usepackage\{amsmath\}
		
		\ \ \ \textbackslash numberwithin\{equation\}\{subsection\}
		
		//...
		
		(equation script...)\\
		\hline
	\end{tabularx}
	\newline\newline
	종속방정식을 라벨링하기 위해서는 \textbf{subequations}환경을 사용하면 된다.\newline
	
	\begin{tabularx}{\textwidth\onehalfspacing}{|X|}
		\hline
		\textbackslash begin\{subequations\}\\
		\ \ \ \ \ \ \textbackslash begin\{align\}\\
		\ \ \ \ \ \ \ \ \ \ \ \ 1+1\&=2\textbackslash\textbackslash\\
		\ \ \ \ \ \ \ \ \ \ \ \ 2\&=1+1\\
		\ \ \ \ \ \ \textbackslash end\{align\}\\
		\textbackslash end\{subequations\}\\
		\hline
	\end{tabularx}
	\begin{subequations}
		\label{eq:align}
		\begin{align}
		1+1&=2\\
		2&=1+1
		\end{align}
	\end{subequations}\clearpage
	\subsubsection{수식 상호 참조}
	\textbf{\textbackslash label}와 \textbf{\textbackslash ref}를 사용하면 손 쉽게 레퍼런싱을 할 수가 있다.\newline
	
	\begin{tabularx}{\textwidth\onehalfspacing}{|X|X|}
		\hline
		\textbackslash begin\{equation\}
		
		\ \ \ \ \ \ \textbackslash label\{eq:sample\}
		
		\ \ \ \ \ \ f(x) = x+1
		
		\textbackslash end\{equation\}
		
		you can reference \textbackslash ref\{eq:sample\} like this
		&\begin{equation}
			\label{eq:sample}
			f(x)=x+1
		\end{equation}
		
		you can reference \ref{eq:sample} like this\\
		\hline
	\end{tabularx}\newline\newline
	또는 다음과 같이 레퍼런싱 할 수 있다.\newline
	
	\begin{tabularx}{\textwidth\onehalfspacing}{|X|X|}
		\hline
		or you can reference like \textbackslash ref\{eq:sample\}
		&or you can reference like \eqref{eq:sample}\\
		\hline
	\end{tabularx}
	\subsection{수식 정렬하기}
	\LaTeX 는 깔끔한 논문을 작성 할 수 있도록 도와준다. 그럼 역시 수식을 깔끔하게 정열하는 방법도 당연히 있을 것이다. 이번 단원에서는 여러문은 \eqref{eq:align}와 같이 수식을 정렬하는 방법을 배울 것이다.\newline
	수식 정열은 \textbf{amsmath}패키지의 \textbf{align}또는 \textbf{align*}(수식 라벨링이 안됨)을 사용한다.\newline
	\textbf{align 환경은 표 안에서 사용이 불가능 함으로 결과는 표 아래에 표시 하겠다.}\newline
	
	\begin{tabularx}{\textwidth\onehalfspacing}{|X|}
		\hline
		\textbackslash begin\{align*\}
		
		\ \ \ \ \ \ f(x) \&=(x+1)(x+2)\textbackslash\textbackslash
		
		\ \ \ \ \ \ \&=x\textasciicircum2+3x+2
		
		\textbackslash end\{align*\}
		\\\hline
	\end{tabularx}
	\begin{align*}
	f(x) &=(x+1)(x+2)\\
	&=x^2+3x+2
	\end{align*}
	위 예시에서 알 수 있듯이 \textbf{\textbackslash\textbackslash}는 줄을 구분할 때에 사용이 되고 \textbf{\&}은 맞출 위치를 지정 할 때에 쓰인다.
	\newline\newline
	\textbf{align}은 다음과 같이도 사용 할 수가 있다.\newline
	
	\begin{tabularx}{\textwidth\onehalfspacing}{|X|}
		\hline
		\textbackslash begin\{align*\}
		
		\ \ \ \ \ \ f(x) \&=1+2+3 \textbackslash nonumber \textbackslash\textbackslash
		
		\ \ \ \ \ \ \&\textbackslash qquad \{\} +4+5
		
		\textbackslash end\{align*\}\\
		\hline
	\end{tabularx}
	\begin{align*}
		f(x) &=11+2+3\nonumber\\
		& \qquad{}+4+5
	\end{align*}\clearpage
	다음과 같은 더 복잡한 수식의 정렬도 가능하다.\newline
	
	\begin{tabularx}{\textwidth\onehalfspacing}{|X|}
		\hline
		\textbackslash begin\{align*\}
		
		\ \ \ \ \ \ f(x) \&=x\textasciicircum2 \& g(x) \&=2\textbackslash\textbackslash
		
		\ \ \ \ \ \ f'(x) \&=2c \& g'(x)\&=0
		
		\textbackslash end\{align*\}\\
		\hline
	\end{tabularx}
	\begin{align*}
	f(x) &=x^2 & g(x)&=2\\
	f'(x) &=2x & g'(x)&=0
	\end{align*}
	\textbf{cases}환경을 사용하면 다음과 같은 수식 정렬도 가능하다.\newline
	
	\begin{tabularx}{\textwidth\onehalfspacing}{|X|X|}
		\hline
		\textbackslash(
		
		f(x)=
		
		\textbackslash begin\{cases\}
		
		x\textasciicircum2+3x+2 \&\text{if} x\textbackslash geq 0
		
		1 \&\text{if} x<0
		
		\textbackslash end\{cases\}
		\textbackslash)
		&
		\[f(x)=
		\begin{cases}
		x^2+3x+2 & \text{if} x \geq 0\\
		1 & \text{if} x <0
		\end{cases}\]
		\\
		\hline
	\end{tabularx}
	\subsection{Formatting Equations}
	여러분은 문서를 작성 할 때에 수식을 강조를 할 때가 있을 것이다. \LaTeX 에서는 수식에 여러가지 색, 폰트, 크기를 적용 할 수 있다.
	\subsubsection{size}
	수식 환경안에서 크기 조절을 하는 스트립트는 다음과 같다.\newline
	
	\begin{tabularx}{\textwidth\onehalfspacing}{|X|X|}
		\hline
		\textbackslash displaystyle asdf&\(\displaystyle asdf\)\\
		\hline
		\textbackslash textstyle asdf&\(\textstyle asdf \)\\
		\hline
		\textbackslash scriptstyle asdf&\(\scriptstyle asdf \)\\
		\hline
		\textbackslash scriptscriptstyle asdf&\(\scriptscriptstyle asdf \)\\
		\hline
	\end{tabularx}
	\clearpage
	\subsubsection{color}
	수식에 색을 넣고 싶다면 \textbf{xcolor}패키지를 이용하여 다음과 같이 사용하면 된다.\newline
	
	\begin{tabularx}{\textwidth\onehalfspacing}{|X|X|}
		\hline
		f(\{\textbackslash color\{red\}\}x)=x-2&\(f({\color{red}x})=x-2\)\\
		\hline
	\end{tabularx}
	\subsubsection{Fonts}
	수식에는 기본으로 제공이 되는 다양한 폰트가 존재한다. 마지막 두 폰트를 제외하고 별다른 패키지 없이 사용 할 수 있다.\newline
	
	\begin{tabularx}{\textwidth\onehalfspacing}{|X|X|}
		\hline
		\textbackslash mathnormal\{asdf1234\}&\(\mathnormal{asdf1234}\)\\
		\hline
		\textbackslash mathrm\{asdf1234\}&\(\mathrm{asdf1234}\)\\
		\hline
		\textbackslash mathit\{asdf1234\}&\(\mathit{asdf1234}\)\\
		\hline
		\textbackslash mathbf\{asdf1234\}&\(\mathbf{asdf1234}\)\\
		\hline
		\textbackslash mathsf\{asdf1234\}&\(\mathsf{asdf1234}\)\\
		\hline
		\textbackslash mathtt\{asdf1234\}&\(\mathtt{asdf1234}\)\\
		\hline
		\textbackslash mathfrank\{asdf1234\}&\(\mathfrak{asdf1234}\)\\
		\hline
		\textbackslash mathcal\{asdf1234\}&\(\mathcal{ASDF}\)\\
		\hline
		\textbackslash mathbb\{asdf1234\}&\(\mathbb{ASDF}\)\\
		\hline
		\textbackslash mathscr\{asdf1234\}&\(\mathscr{ASDF}\)\\
		\hline
	\end{tabularx}\newline\newline
	위의 예시에서 볼 수가 있듯이 밑 3 예시는 영문 대문자만 가능한 폰트이며, 마지막 두 폰트는 순서대로 \textbf{amafonts}와 \textbf{mathrsfs} 패키지를 사용하면 된다.
	\subsubsection{boxed}
	수식에 테두리를 적용하는 방법은 수식 환경 내에서 다음과 같이 스크립트를 입력하면 된다.\newline
	
	\begin{tabularx}{\textwidth\onehalfspacing}{|X|X|}
		\hline
		\textbackslash boxed\{x\textasciicircum2+y\textasciicircum2=4\}&\[\boxed{x^2+y^2=4}\]\\
		\hline
		\multicolumn{2}{|l|}{	\%\textbf{align}환경 안에서는 \textbackslash Aboxed\{\}를 사용하면 된다.}\\
		\hline
	\end{tabularx}\clearpage
	\section{Advanced use of \LaTeX}
	\subsection{자동 조사}
	\LaTeX 에는 자동으로 상호참조 해주는 스크립트가 존재한다. 하지만 한국어로 문서를 작성해야 하는 경우 \textbf{``1과..., 2와...''}같이 숫자에 따라 조사가 다르게 쓰이는 경우가 있다. 여러분은 이럴 때에 다음과 같은 자동 조사기능을 사용하면 추후에 조사를 일일이 검사 해야 하는 문제를 없앨 수 있다.\newline
	자동 조사 명령을 다음과 같다.\newline
	
	\begin{tabularx}{\textwidth\onehalfspacing\large}{|X|}
		\hline
		\textbackslash 이 \textbackslash 가, \textbackslash 을 \textbackslash 를, \textbackslash 와 \textbackslash 과, \textbackslash 로 \textbackslash 으로, \textbackslash 은 \textbackslash 는, \textbackslash 라 \textbackslash 이라\\
		\hline
	\end{tabularx}
	\subsection{참고 문헌 삽입}
	참고문헌을 삽입하는 방법은 \textbf{thebibliography}환경에 다음과 같이 작성하면 인용순 정열을 해준다.\newline
	
	\begin{tabularx}{\textwidth\onehalfspacing}{|X|}
		\hline
		\textbackslash begin\{thebibliography\}\{99\}\\
		\ \ \ \ \ \ \textbackslash bibitem\{example1\} Authors, ``Article name'', Journal name, book, number, \textbf{page}, (year).\\
		\hline
	\end{tabularx}\newline\newline
	필요한 부분에 인용 표기를 하고자 한다면 해당 위치에 다음 스크립트를 입력하면 된다.\newline
	
	\begin{tabularx}{\textwidth\onehalfspacing}{|X|}
		\hline
		\textbackslash cite\{example1\}\\
		\hline
	\end{tabularx}\newline\newline
	bibitem\{\}안의 내용을 기억하기 쉽게, 예로 주저자의 last name과 출판 년도를 사용하면 표기를 한다. \textbf{(ex: bibitem\{choi17\})}
	\clearpage
	\subsection{referencing}
	\subsubsection{labeling}
	여러분은 다음과 같은 환경에 레이블을 달아 다음에 레퍼런싱을 할 때 사용할 수 있다.
	\begin{itemize}
		\item 섹션
		\item 이미지, 표
		\item 수식
		\item 그 외에 캡션을 달수 있는 환경
	\end{itemize}
	수식 라벨링 상호 참조는 \ref{se:eq_label}장에서 이미 다루었다. 하지만, 나머지 요소도 가능하다는 것을 알려주기 위하여 이 단원을 만들었다.\newline
	위와 같이 색션에는 라벨링을 하기 위해서는 색션 내에 아무곳이나 다음 스크립트를 입력하면 된다. 하지만 나중의 편의를 위해서는 색션 바로 다음에 레이블을 추가하는 것을 권장한다.\newline
	
	\begin{tabularx}{\textwidth\onehalfspacing}{|X|}
		\hline
		\textbackslash label\{sec:label\_sample\}\\
		\hline
	\end{tabularx}\newline\newline
	\textbf{나중에 레퍼런싱 할 때 혼동을 최소화 하기 위하여 레이블 앞에 ``sec:''와 같이 어떠한 것을 레이블 한 것인지 명시 해주는 것을 추천한다.}
	
	그 외에 이미지나 표에 레퍼런싱을 하기 위해서는 \textbf{무조건 \textbackslash caption\{\}다음 또는 안에} 위와 같이 레이블을 입력 하면 된다.
	\subsubsection{referencing}
	레퍼런싱은 상호참조를 하고자 하는 위치에 다음 스크립트를 입력하면 된다.\newline
	
	\begin{tabularx}{\textwidth\onehalfspacing}{|X|}
		\hline
		\textbackslash ref\{sam:sample\}\\
		\hline
	\end{tabularx}
	\clearpage
	\subsection{algorithms}
	정보와 같은 과목의 시행절차는 다음과 같이 알고리즘 시리즈로 나열하면 전하고자 하는 알고리즘을 효과적으로 전달을 할 수가 있다.\newline
	
	\begin{tabularx}{\textwidth\onehalfspacing}{|X|X|}
		\hline
		\textbackslash begin\{algorithm\}[H]
		
		\ \ \ \textbackslash KwData\{Learnig \TeX\}
		
		\ \ \ \textbackslash KwResult\{Learned \TeX\}
		
		\ \ \ \textbackslash While\{using \TeX\}\{
		
		\ \ \ \ \ \ use what you know\textbackslash;
				
		\ \ \ \ \ \ \textbackslash eIf\{you don't know sth\}\{
				
		\ \ \ \ \ \ \textbackslash eIf\{It is described in this book\}
		
		\ \ \ \ \ \ \ \ \ \{Learn from this book\textbackslash;\}
		
		\ \ \ \ \ \ \ \ \ \{Ask Google\textbackslash;\}
				
		\ \ \ \ \ \ \}\{keep going\textbackslash;\}
		
		\ \ \ \}
		
		\textbackslash end\{algorithm\}
		&
		\begin{algorithm}[H]
			\KwData{Learnig \TeX}
			\KwResult{Learned \TeX}
			\While{using \TeX}{
				use what you know\;
				\eIf{you don't know sth}{\eIf{It is described in this book}{Learn from this book\;}{Ask Google\;}}{keep going\;}
			}
		\end{algorithm}\\
		\hline
	\end{tabularx}\newline\newline
	여기에서는 알고리즘 입력 외에 또 더 중요한 것을 다루고 있다. \TeX 를 사용하면서 모르는 것이 나오는 것은 당연하다. 명령어를 모를 수 있고, 또는 컴파일을 할 때에 나오는 에러가 무엇인지 모를 수도 있다. \textbf{이때는 우리의 친구 Google에게 물어보도록 하자}.
	\subsection{source code}
	우리의 긴 여행의 마지막 \TeX 에서 소스 코드 입력하기 이다. 소스코드를 입력하는데에는 두가지 방법이 있으며 \textbf{listings}패키지를 사용한다. 그 중 첫번째 예시는 다음과 같다.\newline
	
	\begin{tabularx}{\textwidth\onehalfspacing}{|X|}
		\hline
		\textbackslash begin\{tabularx\}\{\textbackslash textwidth\}\{\textbar X\textbar X\textbar\}\\
		\hline
	\end{tabularx}
	\begin{lstlisting}
	\begin{tabularx}{\textwidth}{|X|X|}
	\end{lstlisting}
	\clearpage
	만약 코드 파일이 있다면 다음과 같은 방법을 이용하는 것을 추천한다.\newline
	
	\begin{tabularx}{\textwidth\onehalfspacing}{|X|}
		\hline
		\textbackslash lstinputlisting[language=TeX, firstline=1, lastline=1398]\{example\_source.py\}\\
		\hline
	\end{tabularx}\newline\newline
	이것을 지원하는 언어 목록들은 다음과 같다:
	
	ABAP, ACSL, Ada, Algol, Ant, Assembler, Awk, bash, Basic, C#, \textbf{C++}, \textbf{C}, Caml, Clean, Cobol, Comal, csh, Delphi, Eiffe, Elan, erlang, Euphoria, Fortran, GCL, Gnuplot, Haskell, \textbf{HTML}, IDL, inform, \textbf{Java} JVMIS, ksh, Lisp, Logo, Lua, make, \textbf{Mathematica}, \textbf{Matlab}, Mercury, MetaPost, Miranda, Mizar, ML, Modelica, Modula-2, MuPAD, NASTRAN, Oberon-2, \textbf{Objective C} , OCL, Octave, Oz, Pascal, Perl, PHP, PL/I, Plasm, POV, Prolog, Promela, \textbf{Python}, R, Reduce, Rexx, RSL, Ruby, S, SAS, Scilab, sh, SHELXL, Simula, SQL, tcl, \textbf{TeX}, VBScript, Verilog, VHDL, VRML, XML, XSLT.
	\section{마치며}
	이 책은 여러분이 \TeX 스크립트와 pdf 사이를 왔다갔다 하면서 공부 또는 작업하기보다, 왼쪽에 코드, 오른쪽에 결과로 예시를 작성하여 이 책만으로도 \TeX 공부를 할 수 있게 하고자 제작이 되었다. 책의 내용을 추가하는 것은 언제나 환영이다. 
\end{document}
